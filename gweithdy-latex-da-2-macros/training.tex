% BEGIN preamble
\svnidlong
{$HeadURL: file:///mnt/between/svn/dysgu/trunk/DoctoralAcademy/gweithdai-latex/gweithdy-latex-da-2-macros/training.tex $}
{$LastChangedBy: cfrees $}
{$LastChangedRevision: 7395 $}
{$LastChangedDate: 2018-02-20 17:36:07 +0000 (Maw, 20 Chw 2018) $}

\input{config/macros-da.cfg}

\title{\LaTeX{} II: Macros}
\subtitle{Based in part on materials produced by UK-TUG Volunteers}
\date{\cymraeg{Chwefror} 21 February 2018}

% END preamble

\begin{document}

\begin{frame}
  \titlepage
\end{frame}

\maketitle

\tutornote{Time: 10:05}

\tableofcontents

\mode<presentation>%
{
  \begin{frame}{Outline}
	\tableofcontents
  \end{frame}
}

\section{Introduction}

\begin{frame}{\LaTeX{}: Key features}

  \begin{itemize}
	\item \TeX{} is a typesetting application.
	\item \TeX{} uses \emph{primitives} to determine how to put text on a page.
	\item \LaTeX{} is  a \emph{format} built on top of \TeX{}.
	\item \LaTeX{} can be used for everything from a one page letter to a 1000~page textbook.
	\item By separating input from output, reusing material becomes much easier.
	\item By separating \emph{content} from \emph{formatting}, we can create flexible, consistent documents which are easier to maintain and modify.
	\item We are using the \emph{engine} pdf\TeX{}.
	Xe\TeX{} and Lua\TeX{} are modern alternatives.
  \end{itemize}

\end{frame}

\overleaf

\section{Logical structure}\tutornote{10:15}

\begin{frame}[fragile]{Logical mark up}

  We are familiar with the importance of using \emph{logical} mark up.

  For example,
  \begin{itemize}
	\item \cs{title}\marg{\meta{title}}, \cs{author}\marg{\meta{author}} \& \cs{maketitle}
	\item \cs{section}\oarg{\meta{short title}}\marg{\meta{title}} \& \cs{subparagraph}\oarg{\meta{short title}}\marg{\meta{title}}
	\item \cs{emph}\marg{\meta{text}}
	\item \cs{label}\marg{\meta{key}} \& \cs{ref}\marg{\meta{key}}
  \end{itemize}

\end{frame}

\begin{frame}[fragile]{Why use logical mark up?}
  Logical mark up is \emph{semantic}.

  It reflects a functional role within the document.
  For example,
  \begin{itemize}
	\item \cs{title}\marg{\meta{title}} indicates that \meta{title} is the title of the document.
	\item \cs{section}\oarg{\meta{short title}}\marg{\meta{title}} tells \LaTeX{} to begin a new section with the title \meta{title} and optionally specifies a shortened title, \meta{short title} as well.
	\item \cs{ref}\marg{\meta{key}} is a reference to a particular place in the text.
  \end{itemize}

\end{frame}

\begin{frame}[fragile]{Why extend logical mark up?}
  \begin{block}{Example}
	\begin{verbatim}
	  \section{A section about food additives}\label{wombles}
	  ...
	  As argued in section \ref{wombles}.
	\end{verbatim}
  \end{block}
  This will work but it is not ideal.
  \begin{itemize}
    \item Our choice of ‘wombles’ is arbitrary.
	It would be better to use an appropriate label e.g.~‘food-additives’ or ‘fadd’.
	\item Having to write out ‘section’ is non-ideal.
	\begin{itemize}
	  \item We must remember that \cs{label}\marg{wombles} is a section rather than something else.
	  \item If we want to indicate that a reference is to a section using ‘\S’ or ‘[Tt]opic’, we have to locate and modify every occurrence.
	\end{itemize}
  \end{itemize}
\end{frame}

\begin{frame}[fragile]{Extending logical mark up with packages}
  Packages extend \LaTeX{} in various ways and serve many purposes.

  One purpose is the \emph{extend logical mark up}.
\end{frame}
\begin{frame}{\pkg{fancyref}: More logical cross-referencing}
  \alert<1>{\pkg{fancyref}} enhances \LaTeX's commands for cross-referencing.

  The syntax follows this pattern:
  \begin{semiverbatim}
    \cs{label}\marg{\alert<2>{\meta{prefix}:}\meta{key}} \& \alert<1>{\cs{fref}}\marg{\alert<2>{\meta{prefix}:}\meta{key}}
  \end{semiverbatim}
  \pkg{fancyref} provides:
  \begin{itemize}
    \item a set of pre-defined \alert<2>{\meta{prefix}} values;
    \item an extensible interface i.e.~you can customise the effects, modify the \alert<2>{\meta{prefix}} values and define entirely new categories of cross-reference.
  \end{itemize}
  This is particularly useful in large projects such as books, dissertations and theses.
  \alert<3>{\pkg{cleveref}} is another option.
\end{frame}

\begin{frame}[fragile,plain]
\begin{block}{Example input}
  \begin{verbatim}
	\usepackage{fancyref}
	...
	\section{A section about food additives}\label{sec:fadd}
	...
	\Fref{sec:fadd} raised several concerns about food additives:
	\begin{enumerate}
	  \item \label{enum:e-nos}`E numbers'
	  \item \label{enum:health}Health impact
	\end{enumerate}
	\Fref{enum:health} is discussed in \fref{subsec:health}.
	In \fref{subsec:e-numbers}, we explore so-called E numbers.
	\subsection{E numbers}\label{subsec:e-numbers}
	...
	\subsection{Health impacts}\label{subsec:health}
  \end{verbatim}
  \vskip -\baselineskip
\end{block}
  Remember: resolving cross-references needs \emph{two} \LaTeX{} runs.
\end{frame}

\tutornote{Mention \pkg{cleveref}}

\begin{exercise}
  Try producing the following document.
  \verbatiminput{examples/example10}

  Experiment with \pkg{fancyref} cross-references to sections, subsections and items in ordered lists.

  What happens if you replace \cs{usepackage}\marg{fancyref} with \cs{usepackage}\oarg{german}\marg{fancyref}?
  What happens if you use \cs{usepackage}\oarg{english}\marg{fancyref}?

  What happens if you replace \cs{ref}\marg{apd:sample} by \cs{fref}\marg{apd:sample}?

  Try adding the following to your preamble:
  \begin{verbatim}
	\fancyrefaddcaptions{english}{%
	  \newcommand*{\Frefappendixname}{\appendixname}%
	  \newcommand*{\frefappendixname}{%
		\MakeLowercase{\Frefappendixname}}}
	\newcommand*{\fancyrefappendixlabelprefix}{apd}
	\frefformat{vario}{\fancyrefappendixlabelprefix}{{\frefappendixname}\fancyrefdefaultspacing#1}
	\Frefformat{vario}{\fancyrefappendixlabelprefix}{{\Frefappendixname}\fancyrefdefaultspacing#1}
	\frefformat{plain}{\fancyrefappendixlabelprefix}{{\frefappendixname}\fancyrefdefaultspacing#1}
	\Frefformat{plain}{\fancyrefappendixlabelprefix}{{\Frefappendixname}\fancyrefdefaultspacing#1}
  \end{verbatim}

  How would you need to modify this if you were using \verb|german|?

  Can you create an ordered list using \cs{label}\marg{issue:\meta{key}} such that \cs{fref}\marg{issue:\meta{key}} produces ‘issue \meta{number}’?

\end{exercise}

\begin{exercise}
  If time permits, choose a package which interests you from the list provided.

  Find the documentation for the package on \url{ctan.org}.

  Try integrating some simple functionality from the package into your document.
  Don't forget to add \cs{usepackage}\marg{\meta{package name}} to your preamble!
\end{exercise}


\section{Creating your own macros}\tutornote{10:45}


\begin{frame}{Why create your own macros?}
  \begin{itemize}
  	\item You want to extend logical mark up further.
	\begin{itemize}
	  \item Increase the consistency of formatting.
	  \item Reduce errors and make debugging easier.
	  \item Increase flexibility.
	  \item Decrease typing!
	\end{itemize}
	\item You want to do something new.
	\item You want to do something better.
	\item You want to do something differently.
  \end{itemize}
\end{frame}

\begin{frame}<1-| handout:1-2>{How many macros should you make?}
  \mode<article>{
    Standard question:
    \begin{quotation}
      How many {packages} should I {load}?
    \end{quotation}

    {Standard} answer:
    \begin{quotation}
      No more than you need and no fewer than you need.
    \end{quotation}
  }
  \alt<2-| handout:2>{Macro}{Standard} question:
    \begin{quotation}
      How many \alt<2-| handout:2>{macros}{packages} should I \alt<2-| handout:2>{create}{load}?
    \end{quotation}

    \alt<2-| handout:2>{Macro}{Standard} answer:
    \begin{quotation}
        No more than you need and no fewer than you need.
    \end{quotation}
\end{frame}

\begin{frame}{Guidelines for your own macros}{Don't\dots}
  \begin{block}{Reduce, Reuse, Recycle (Don't do unnecessary work.)}
	\onslide<+->
	\lettrine{R}{educe:}
	  Avoid creating macros you do not need.\newline
	  Avoid creating macros as short-cuts for standard macros.
      \begin{itemize}
        \item[e.g.]  Don't create a command \cs{sn} to save typing \cs{section}.
      \end{itemize}

	\onslide<+->
	\lettrine{R}{euse:}
	  Don't repeat yourself.\newline
	  Create flexible macros you can use in many contexts.\newline
	  If you want to use a set of macros in many different documents, make a separate file and load that file in each document.

	\onslide<+->
	\lettrine{R}{ecycle:}
	  Avoid creating more macros than you need.\newline
	  Could one or two macros with different options replace a dozen less flexible ones?
  \end{block}
\end{frame}

\begin{frame}{Guidelines for your own macros}{Do\dots}
  \begin{block}{Logical, Meaningful, Explicable}
	\onslide<+->
	\lettrine{L}{ogical:} Do create macros which increase logical mark up.\newline
	  Macros should make it easier to modify layout and formatting.
	  \begin{itemize}
        \item[e.g.]  These slides use \cs{pkg}\marg{\meta{package name}} for package names.
      \end{itemize}

	\onslide<+->
	\lettrine{M}{eaningful:} Do use macro names which make their purpose clear.\newline
	  Forgotten macros should be easy to maintain when rediscovered.
	  \begin{itemize}
        \item[e.g.]  \cs{marg}\marg{\meta{mandatory argument}} is used for mandatory arguments.
      \end{itemize}

	\onslide<+->
	\lettrine{E}{xplicable:} Do use comments to explain your code.\newline
	  This should cover usage and implementation.
  \end{block}
\end{frame}

\begin{frame}[fragile]{What are macros?}
	‘Macros’ are commands, environments etc.

	You are already familiar with the way \LaTeX{} commands work:
	\begin{semiverbatim}
		\cs{noarguments}
		\cs{argumentrequired}\marg{\meta{mandatory argument}}
		\cs{argumentoptional}\oarg{\meta{optional argument}}
		\cs{morecomplex}\oarg{\meta{optional argument}}\marg{\meta{mandatory argument}}
		  \marg{\meta{mandatory argument}}\dots
		\cs{starred*}\oarg{\meta{optional arguments}}\marg{\meta{mandatory arguments}}
	\end{semiverbatim}
	You are also familiar with \LaTeX{} environments:
	\begin{semiverbatim}
		\cs{begin}\marg{\meta{environment}}\oarg{\meta{optional argument}}
		  \marg{\meta{mandatory argument}}
		  \dots
		\cs{end}\marg{\meta{environment}}
	\end{semiverbatim}
\end{frame}

\begin{frame}{Creating your own commands}

  \begin{block}{Syntax}
    \alert<1>{\cs{newcommand*}}\alert<2>{\marg{\cs{\meta{name}}}}\alert<3>{\marg{\meta{replacement text}}}\newline
    \alert<1>{\cs{newcommand*}}\alert<2>{\marg{\cs{\meta{name}}}}\alert<4>{\oarg{\meta{number}}}%
    \alert<3>{\marg{\meta{replacement text}}}
  \end{block}

  \begin{block}{Examples}
	\cs{newcommand*}\marg{\cs{authorname}}\marg{Joseph Wright} \newline
	\cs{newcommand*}\marg{\cs{important}}\oarg{1}\marg{\cs{textbf}\marg{\#1}}
  \end{block}

\end{frame}

\begin{exercise}
  Can you think why it might make sense to create a command \cs{authorname} even though \LaTeX{} already provides \cs{author}\marg{\meta{name}}?

  Try creating your own simple commands using \cs{newcommand}.
  Think of how to create commands using 1, 2 and 3 arguments.

  Define a new command as follows:
  \begin{verbatim}
	\newcommand*{\veryimportanttext}[1]{\begin{center}\bfseries #1\end{center}}
  \end{verbatim}

  Can you use the command to produce the following?
  \begin{center}
	\bfseries This is a very important piece of text.
  \end{center}
  What about this?
  \begin{center}
  	\bfseries This is a very important piece of text.

	This piece of text, however, is even more important.
  \end{center}
  What is the difference?
  Does removing the asterisk from \cs{newcommand*} help?
\end{exercise}

\tutornote{11:10}

\begin{frame}{Extended syntax}

  \begin{block}{‘Short’ and ‘long’ arguments}
  	A ‘short’ argument cannot include a paragraph break.\\
	A ‘long’ argument can include one.

    Use \alert<2>{\cs{newcommand*}} if all arguments must be \alert<2>{short}.\\
    Use \alert<3>{\cs{newcommand}} if some arguments may be \alert<3>{long}.
  \end{block}

  \begin{block}{Examples}
	\cs{newcommand*}\marg{\cs{important}}\oarg{1}\marg{\cs{textbf}\marg{\#1}}\newline
	\cs{newcommand}\marg{\cs{important}}\oarg{1}\marg{\{\cs{bfseries} \#1\}}
  \end{block}

\end{frame}

\begin{frame}{Optional arguments}

  Sometimes, you might want to allow an optional argument.

  \begin{block}{Optional arguments}
  	\cs{newcommand*}\marg{\cs{\meta{name}}}\oarg{\meta{number}}%
    \alert<1>{\oarg{\meta{default value of optional argument}}}\marg{\meta{replacement text}}\newline
  	\cs{newcommand}\marg{\cs{\meta{name}}}\oarg{\meta{number}}%
    \alert<1>{\oarg{\meta{default value of optional argument}}}\marg{\meta{replacement text}}
  \end{block}

  If there is an optional argument, it must be the \textbf{first}.

  \begin{block}{Examples}
	\cs{newcommand*}\marg{\cs{veryimportant}}\oarg{1}\oarg{Attention!}\marg{\cs{textbf}\marg{\#1}}\newline
	\cs{newcommand*}\marg{\cs{aside}}\oarg{2}\oarg{Note}\marg{\cs{textbf}\marg{\#1: \#2}}
  \end{block}

\end{frame}

\begin{exercise}
  Suppose that you want to be able to say \cs{person}\oarg{\meta{phone}}\marg{\meta{first name}}\marg{\meta{surname}}.
  You would like this to produce the following output when the optional command is provided:
  \begin{quotation}
	\textbf{\meta{surname}}, \meta{first name} \meta{phone}
  \end{quotation}
  or otherwise:
  \begin{quotation}
	\textbf{\meta{surname}}, \meta{first name}
  \end{quotation}
  Can you create a command which behaves as you wish?

  Suppose that you want to change the output of \cs{person} in the second half of your document.
  This time, you would like the output to be:
  \begin{quotation}
	\textbf{\meta{first name}} \meta{surname} \meta{phone}
  \end{quotation}
  or:
  \begin{quotation}
	\textbf{\meta{first name} \meta{surname}}
  \end{quotation}

  What happens if you try to use \cs{newcommand*} to make this change?

  What difference does it make if you use \cs{renewcommand*} instead?

  Can you figure out what \cs{providecommand*} does?

  (Check appendix \ref{subsec:commands}.
  Were you right?)
\end{exercise}

\tutornote{11:25}

\begin{frame}{Creating your own environments}

  \begin{block}{Syntax}
	\emph{All arguments must be short:}\smallskip

    \alert<1>{\cs{newenvironment*}}\alert<3>{\marg{\meta{name}}}\alert<5>{\marg{\meta{begin env}}}\alert<6>{\marg{\meta{end env}}}

    \alert<1>{\cs{newenvironment*}}\alert<3>{\marg{\meta{name}}}\alert<4>{\oarg{\meta{number}}\oarg{\meta{default value}}}%
    \alert<5>{\marg{\meta{begin env}}}\alert<6>{\marg{\meta{end env}}}\medskip

	\noindent\emph{Some argument may be long:}\smallskip

    \alert<2>{\cs{newenvironment}}\alert<3>{\marg{\meta{name}}}\alert<5>{\marg{\meta{begin env}}}\alert<6>{\marg{\meta{end env}}}

    \alert<2>{\cs{newenvironment}}\alert<3>{\marg{\meta{name}}}\alert<4>{\oarg{\meta{number}}\oarg{\meta{default value}}}%
      \alert<5>{\marg{\meta{begin env}}}\alert<6>{\marg{\meta{end env}}}\\
  \end{block}

\end{frame}
\begin{frame}[fragile]

  \begin{block}{Examples}
	\begin{verbatim}
	\newenvironment*{inset}{%
		  \begin{minipage}{.75\textwidth}%
	}{%
		  \end{minipage}%
	}
	\end{verbatim}
	\begin{verbatim}
	\newenvironment*{titledinset}[1]{%
		  \begin{minipage}{.75\textwidth}%
		  \textbf{#1}\medskip\par
	}{%
		  \end{minipage}%
	}
	\end{verbatim}
  \end{block}

\end{frame}

\begin{exercise}
	Experiment by creating some simple environments of your own.

	Can you define the environment \env{titledinset} so that the title is centred but the body of the environment is not?

	\LaTeX{} also provides \cs{renewenvironment} and \cs{renewenvironment*} for redefining custom environments.

	Can you create an environment which would be suitable for typesetting acknowledgements in a thesis or dissertation?
	Use \cs{clearpage} to ensure that your acknowledgements appear on a page of their own.
	Use \cs{cleardoublepage} to ensure that they appear on a right-hand page of their own in a double-sided document.
\end{exercise}

\Cref{sec:macros} provides a more detailed overview of the facilities \LaTeXe{} offers for (re)defining macros.
\Cref{sec:beyond2e} alludes to the wider world beyond.

\tutornote{11:40}

\begin{frame}{Creating your own packages}

  It is convenient to put the commands and environments you create into your own \LaTeX{} package.
  \begin{itemize}
  	\item You can reuse the macros in many documents.
	\item Macro definitions do not distract from content when editing your document.
	\item Your macros can be easily shared with others.
  \end{itemize}

\end{frame}

\begin{frame}[fragile]{‘Style’ files}

  \LaTeX{} packages are provided as ‘style’ files with extension \texttt{.sty}.
  \begin{block}{Syntax}
	\begin{verbatim}
		\NeedsTeXFormat{LaTeX2e}
		\ProvidesPackage{<name>}% filename without extension .sty

		\RequirePackage{<name>}% load a package like this
		\RequirePackage[<options>]{<name>}% or like this

		<macro definitions>

		\endinput
	\end{verbatim}
  \end{block}

\end{frame}

\begin{exercise}
	Create a package for the commands and environments you defined earlier.
	Save it as part of your project in Overleaf and put \cs{usepackage}\marg{\meta{name}} in the preamble of your \texttt{.tex} file, deleting the macro definitions you've transferred to the package.
	Check that your document still compiles as it should!
\end{exercise}

\tutornote{12:00: end}

\mode
<all>

\furtherinfo*[%
  custom macros:
  \begin{itemize}
    \item Appendix \ref{handouts:sec:macros}: summary and extension of today's material on creating custom macros in \LaTeXe.
    \item Appendix \ref{handouts:sec:beyond2e}: brief overview of custom macro creation with \pkg{xparse}, \LaTeX 3 \& \TeX.
  \end{itemize}%
]
\mode
<article>

\appendix\label{app}

\section<1-| beamer:0>{Creating Macros}\label{sec:macros}
% BEGIN sec:macros

\LaTeX{} macros for defining custom commands and environments include:
\begin{itemize}
  \item \cs{newcommand}
  \item \cs{newcommand*}
  \item \cs{renewcommand}
  \item \cs{renewcommand*}
  \item \cs{providecommand}
  \item \cs{providecommand*}
  \item \cs{newenvironment}
  \item \cs{newenvironment*}
  \item \cs{renewenvironment}
  \item \cs{renewenvironment*}
\end{itemize}
It is best to start with these since you should prefer them to \TeX{} macros such as \cs{def}.
However, sometimes only the \TeX{} macros will work.
It is easier to understand when and why \cs{def} and friends are needed once you have a good understanding of how the \LaTeX{} macros work.

\subsection{Defining, redefining and providing commands}\label{subsec:commands}
% BEGIN subsec:commands

This section explains how to define and redefine \emph{commands}.
Commands defined using the \LaTeXe{} facilities listed above may take
\begin{itemize}
  \item no arguments (e.g.~\cs{maketitle});
  \item one or more mandatory arguments (e.g.~\cs{textbf}\marg{\meta{text}});
  \item an optional argument (e.g.~\cs{pagebreak}\oarg{\meta{number}});
  \item an optional argument combined with one or more mandatory arguments (e.g.~\cs{title}\oarg{\meta{short title}}\marg{\meta{title}}).
\end{itemize}
The syntax for defining a \emph{new} command is
\simp{%
  \cs{newcommand}\marg{\cs{\meta{name}}}\oarg{\meta{n}}\oarg{\meta{default}}\marg{\meta{definition}}%
}
where \meta{n} is the total number of arguments and \meta{definition} is the definition of the command.
If \meta{n} is not specified, \LaTeX{} assumes the command cannot take arguments.
If \meta{default} is given, the first argument is optional and, if missing, \meta{default} is used.
The definition of the command may refer to the arguments, if applicable, as \narg, \dots, \narg[9].

For example, this is how \cs{tableofcontents} is defined in \filename{article.cls}:

\begin{verbatim}
  \newcommand\tableofcontents{%
    \section*{\contentsname
      \@mkboth{%
        \MakeUppercase\contentsname}{\MakeUppercase\contentsname}}%
    \@starttoc{toc}%
  }
\end{verbatim}
The command takes no arguments, either mandatory or optional, so only the macro name, \cs{tableofcontents} and the definition of the macro are given.

Here is an example with a single, mandatory argument from \filename{ltnews.cls}:
\begin{verbatim}
    \newcommand\cs[1]{\texttt{\textbackslash#1}}
\end{verbatim}
The \narg{} refers to the first argument.

This is an example from \filename{latex.ltx} defining a new command which takes 2 arguments, one mandatory and one optional.
\begin{verbatim}
  \newcommand\qbezier[2][0]{\bezier{#1}#2}
\end{verbatim}
\narg{} refers to the first, optional argument and \narg[2] to the second, mandatory argument.
If the optional argument is not given, the default value of \textt{0} will be used as \narg.
That is, the following are equivalent:
\begin{verbatim}
  \qbezier[0]{...}
  \qbezier{...}
\end{verbatim}


The number of arguments may not exceed 9 and no more than one optional argument may be specified using \cs{newcommand}.
All other arguments must be mandatory.
Moreover, if an optional argument is available, it must be specified first and the syntax of the defined command is always
\simp{\cs{mycommand}\oarg{\meta{optional}}\marg{\meta{mandatory}}\marg{\meta{mandatory}}\marg{\meta{mandatory}}\textt{\dots}.}

\cs{newcommand} checks for an existing command of the same name and produces an error if it finds one.
So there is a built-in check to prevent accidentally overwriting an existing command.

\cs{newcommand*} is just like \cs{newcommand} but its arguments cannot contain paragraph breaks.

To \emph{redefine} an existing command, you can use \cs{renewcommand} or \cs{renewcommand*}.
In this case, \LaTeX{} checks whether or not the command already exists and produces an error if it does not.

If you want to \emph{define a command only if it does not exist already}, you can use \cs{providecommand} or \cs{providecommand*}.
% END subsec:commands

\subsection{Defining and redefining environments}\label{subsec:env}
% BEGIN subsec:env

This section explains how to define and redefine \emph{environments}.
Environments defined using the \LaTeXe{} facilities listed above may take
\begin{itemize}
  \item no arguments e.g.~\env{document}
  \begin{semiverbatim}
    \cs{begin}\marg{document} \dots{} \cs{end}\marg{document}
  \end{semiverbatim}
  \item one or more mandatory arguments e.g.~\env{thebibliography}
  \begin{semiverbatim}
    \cs{begin}\marg{thebibliography}\marg{\meta{widest label}} \dots{} \cs{end}\marg{thebibliography}
  \end{semiverbatim}
  \item an optional argument e.g.~\env{figure}
  \begin{semiverbatim}
    \cs{begin}\marg{figure}\oarg{\meta{permitted location specifications}} \dots{} \cs{end}\marg{figure}
  \end{semiverbatim}
  \item an optional argument combined with one or more mandatory arguments e.g.~\env{minipage}
  \begin{semiverbatim}
    \cs{begin}\marg{minipage}\oarg{\meta{alignment}}\marg{\meta{width}} \dots{} \cs{end}\marg{minipage}
  \end{semiverbatim}
\end{itemize}
The syntax for defining a \emph{new} environment is
\simp{%
  \cs{newenvironment}\marg{\meta{env name}}\oarg{\meta{n}}\oarg{\meta{default}}\marg{\meta{begin env}}\marg{\meta{end env}}%
}
where \meta{env name} is the name of the new environment, \meta{n} is the number of arguments (0 if not specified) and \meta{default} is the default value of the first argument if the first argument is optional (all arguments are mandatory if this is not specified).
\meta{begin env} and \meta{end env} specify the code to be executed at the beginning and end of the environment.

This is very like \cs{newcommand}.
\meta{n} cannot exceed 9, only one argument may be optional and it must be the first argument, and \LaTeX{} will complain if an environment of the same name already exists. \narg,\dots \narg[9] refer to the arguments given to the environment if applicable.

Here is a (complicated!) example from \filename{cuisine.sty} with 3 arguments, all mandatory:

\begin{verbatim}
  \newenvironment{recipe}[3]{%
    \stepcounter{r@cipenumber}
    \let\newstep\m@thodend
    \let\recipen@wpage\newpage
    \let\newpage\r@cipen@wpage
    \let\0\d@grees
    \let\degrees\d@grees
    \let\fr\fr@ction
    \let\ing\ingr@dient
    \let\ingredient\ingr@dient
    \let\freeform\fr@eform
    \n@wpagingfalse%
    \setlength{\parindent}{0pt}
    \savebox{\st@pingrbox}[\R@cipeIQUWidth]{}
    \savebox{\st@pmethodbox}[\R@cipeMethodWidth]{}
    \ifind@xing
    \addcontentsline{toc}{subsection}{#1}
    \fi
    \r@cipetitle{#1}{#2}{#3}
    \vskip0.2cm%
    \p@stingred%
  }%
  {%
    \pr@ingred%
    \ifnum\value{st@pnumber}=0%  then complain!
    \PackageWarning{cuisine}{The recipe did not have any steps}%
    \fi%

    \pagebreak[0]%
    \medskip%
    \@endpetrue%
  }%
\end{verbatim}

\cs{newenvironment*} is just like \cs{newenvironment} but its arguments cannot contain paragraph breaks.

To redefine an existing environment, use \cs{renewenvironment} or \cs{renewenvironment*}.
Again, \LaTeX{} will complain in this case if the environment does not already exist.
% END subsec:env

\subsection{Counting \& measuring}\label{subsec:cntmsr}

% BEGIN subsec:cntmsr

Sometimes you need to count or measure things and \LaTeX{} allows you to create custom counters and lengths for these purposes.

The syntax for defining a new counter is
\simp{%
  \cs{newcounter}\marg{\meta{countername}}%
}
To set a counter to a given value,
\simp{%
  \cs{setcounter}\marg{\meta{countername}}\marg{\meta{integer}}%
}
To add to a counter,
\simp{%
  \cs{addtocounter}\marg{\meta{countername}}\marg{\meta{integer}}%
}
To increment a counter by 1,
\simp{%
  \cs{stepcounter}\marg{\meta{countername}}%
}
To increment a counter by 1 and enable \cs{label} and \cs{ref} to use the resulting value,
\simp{%
  \cs{refstepcounter}\marg{\meta{countername}}%
}

The syntax for defining a new length is
\simp{%
  \cs{newlength}\marg{\cs{\meta{lengthname}}}%
}
To set a length to a given dimension,
\simp{%
  \cs{setlength}\marg{\meta{lengthname}}\marg{\meta{dimension}}%
}
To add to a length,
\simp{%
  \cs{addtolength}\marg{\meta{lengthname}}\marg{\meta{dimension}}%
}
To set a length to the width, height or depth of something,
\begin{itemize}
  \item \cs{settowidth}\marg{\meta{lengthname}}\marg{\meta{something whose width is to be measured}}
  \item \cs{settoheight}\marg{\meta{lengthname}}\marg{\meta{something whose height is to be measured}}
  \item \cs{settodepth}\marg{\meta{lengthname}}\marg{\meta{something whose depth is to be measured}}
\end{itemize}

% END subsec:cntmsr

\subsection{Internal \protect\emph{vs.}\ external}\label{subsec:intext}

% BEGIN subsec:intext

You will sometimes see the following macros in examples:
\begin{itemize}
  \item \cs{makeatletter}
  \item \cs{makeatother}
\end{itemize}
Usually, \verb|@| is not permitted in macro names in documents.
However, it is used extensively in the names of macros defined in classes, packages etc\@.
In general, \verb|@| marks a macro as an \emph{\bfseries internal} one and not intended to be used by end-users in documents.
Typically, such macros are more liable to change in future development, because authors may change the internal implementation of a package or class, while keeping the user-interface stable.
Hence, these should not be redefined willy-nilly: if there is a supported way to do what you need without redefining internals, that should be used instead.

If you do redefine these macros --- or if you use \verb|@| to mark your own macros as internal --- you will need to know the following.
\begin{itemize}
  \item In a class (\verb|.cls|) or package {\verb|.sty|} file, \verb|@| may be used freely in macro names.
  \item In a document (\verb|.tex|), it is necessary to used
  \begin{verbatim}
    \makeatletter % allow use of @ in macro names
    ...
    \makeatother % return to normal state
  \end{verbatim}
  This should generally be avoided in the body of the \env{document}.
  If you need to use this at all, use in in the preamble, before the \env{document} itself begins.
\end{itemize}

% END subsec:intext

% END sec:macros

\section<1-| beamer:0>{Beyond \LaTeXe}\label{sec:beyond2e}

% BEGIN sec:beyond2e

When you are comfortable with these methods of customisation, you will find that there are limitations on what you can do.
At this stage, you will want to take a look at \pkg{xparse} which provides much more powerful and flexible options from the work developing \LaTeX 3.
For further development, you may wish to look at the underlying \LaTeX 3 syntax, which is really, really nice.
In particular, it is far more logical and consistent than the syntax used by \TeX{} and even \LaTeX{}.
But these methods will make more sense once you are familiar with the possibilities offered by the \LaTeXe{} macros.

\subsection[xparse]{\protect\pkg{xparse}}\label{subsec:xparse}

% BEGIN subsec:xparse

The standard \LaTeXe{} facilities for defining and redefining macros support a limited syntax\footnote{%
  The reality is a little more complicated than this, but for current purposes, the description here is a good enough approximation.%
}:
\begin{itemize}
  \item no possibility to define a starred variant;
  \item only one optional argument allowed;
  \item any optional argument must be the first one;
  \item optional arguments must be given a default value;
  \item optional arguments must use the \oarg{\meta{value}} syntax;
  \item mandatory arguments must use the \marg{\meta{value}} syntax.
\end{itemize}
\pkg{xparse} removes these limitations, along with a few others.
For a full description of the macro creation framework provided, see the package's manual.
These include support for commands, environments, argument testing and argument processing.
Here, I'm just going to give an illustrative example, together with a subset of the syntax.

\begin{itemize}
  \item \cs{NewDocumentCommand}\cs{\meta{name}}\marg{\meta{argument specification}}\marg{\meta{definition}}
\end{itemize}
\cs{NewDocumentCommand} corresponds to \cs{newcommand}.
{\tikzset{external/export=false}%
  \begin{figure}
    \centering
    \begin{tikzpicture}[remember picture]
      \node (spec) [font=\ttfamily] { \dots{} \marg{ \subnode{s}{s} \subnode{o}{o} \subnode{sm}{m} \subnode{O}{O} \subnode{O default}{\marg{1}} \subnode{sum}{+}\subnode{d}{d} \subnode{d delim}{()} \subnode{plus}{+}\subnode{m}{m} } \dots };
      \begin{scope}[Latex-Circle, shorten <=-1pt]
        \draw (s.south) [bend left] to +(-10mm,-10mm) node (star) [anchor=east] {optional star};
        \node (bool) [inner ysep=0pt, anchor=north east] at (star.south east) {(Boolean)};
        \draw (o.south) [bend left] to  ([yshift=-5mm]bool.south east) node (opt) [anchor=east] {optional argument};
        \node (no default) [inner ysep=0pt, anchor=north east] at (opt.south east) {(no default)};
        \draw (sm.south) [bend left] to  ([yshift=-5mm]no default.south east) node (short) [anchor=east] {mandatory argument};
        \draw (O.south) [bend left] to ([yshift=-5mm]short.south east) node (optional) [anchor=east] {optional argument};
        \node (default) [inner ysep=0pt, anchor=north east] at (optional.south east) {(with default)};
        \draw (O default.south) [bend left] to ([yshift=-5mm]default.south east) node [anchor=east] {default value for \texttt{O}};
        \draw (m.south) [bend right] to +(10mm,-10mm) node (mandatory) [anchor=west] {mandatory argument};
        \draw (plus.south) [bend right] to ([yshift=-5mm]mandatory.south west) node (long) [anchor=west] {\texttt{m} is ‘long’};
        \node (pars) [anchor=north west, inner ysep=0pt] at (long.south west) {(can include paragraph breaks)};
        \draw (d delim.south) [bend right] to ([yshift=-5mm]pars.south west) node (delimiters) [anchor=west] {delimiters for \texttt{d}};
        \draw (d.south) [bend right] to ([yshift=-5mm]delimiters.south west) node (delimited) [anchor=west] {delimited optional argument};
        \node (d delims) [anchor=north west, inner ysep=0pt] at (delimited.south west) {(non-standard delimiters)};
        \draw (sum.south) [bend right] to ([yshift=-5mm]d delims.south west) node (long d) [anchor=west] {\texttt{d} is ‘long’};
        \node [anchor=north west, inner ysep=0pt] at (long d.south west) {(can include paragraph breaks)};
      \end{scope}
      \end{tikzpicture}
      \caption[An argument specification for an xparse macro-creation command.]{An argument specification for an \pkg{xparse} macro-creation command.}\label{fig:argspec}
    \end{figure}%
}
It will give an error if \cs{\meta{name}} is already defined; otherwise it will define it as instructed.
However, the \meta{argument specification} provides a lot more flexibility than \cs{newcommand}'s.
The specification consists of a series of letters, possibly preceded by \verb|+| and possibly followed by additional information.
An example of an argument specification is shown in \cref{fig:argspec}.

\pkg{xparse} is part of the \LaTeX 3 project, whose aim is to eventually provide the successor to \LaTeXe.
The package is written using the syntax developed by the project, \pkg{expl3}.
For details, see \filename{interface3,pdf} (\verb|texdoc interface3|).
Note, in particular, the importance of \pkg{expl3}'s recommendations for naming macros.
This makes code written in this syntax much more readable, consistent and predictable than anything offered by \LaTeXe{} or \TeX{} itself.
If you wish to use the support provided by \pkg{expl3}, then you should familiarise yourself with these guidelines and name your own macros accordingly.
In particular, the syntax distinguishes between \emph{functions} and \emph{variables/constants}.
\begin{itemize}
  \item Variable/constant names take the form \verb|\|\meta{l\textbar g\textbar c}\verb|_|\meta{prefix}\verb|_|\meta{name}\verb|_|\meta{type}.
  \verb|l| indicates a \emph{local} variable, \verb|g| a global variable and \verb|c| a constant.
  \meta{prefix} is a distinctive string used for macros provided by a particular package or class.
  This is designed to help reduce the potential for the name conflicts which plague \LaTeXe.
  \meta{name} should be a descriptive name which indicates what the variable or constant is for.
  \meta{type} indicates the type of value the variable/constant holds.
  For example, \verb|\l_alexmacros_chapter_title_tl| might be used to hold the current value of the chapter title in a package \pkg{alexmacros}.
  A chapter title will be a sequence of tokens, hence, \verb|tl| which is the standard suffix indicating a \emph{token list}.
  Other suffixes indicate other kinds of values e.g.~\verb|bool| for Booleans, \verb|dim| for dimensions, \verb|int| for integers and so on.
  \item Function names take the form \verb|\|\meta{prefix}\verb|_|\meta{name}\verb|:|\meta{argument specification}.
  For example, \verb|\alexmacros_chapter_abstract:n| might be a function used to typeset a chapter abstract in \pkg{alexmacros}.
  The \verb|n| after the colon shows that this function takes one argument in curly brackets.
  In \pkg{expl3}'s world, the functions' names tell you how many arguments they take, if any, and which kind of thing each argument should be. 
  These are, of course, intended to be used in class and package code rather than in regular documents.
  To allow the use of \verb|\alexmacros_chapter_abstract:n| in a document, you would use \pkg{xparse} to provide a document-level wrapper.
\begin{verbatim}
\NewDocumentCommand\alexabstract { +m }{\alexmacros_chapter_abstract:n {#1}}
\end{verbatim}
\end{itemize}
Again, for further information, consult \LaTeX 3's documentation: the above is just to give you a sense of what the syntax looks like.

% END subsec:xparse

\subsection{\TeX{} macro creation}\label{subsec:tex}

% BEGIN subsec:tex

Finally, you may wish eventually to look at \cs{def} and friends.
These are \TeX{} commands, as opposed to \LaTeX{} ones.
If possible, consider using the \LaTeX 3 syntax supported by \pkg{expl3}.
This support is not yet complete, so there are still some things which you can only do using the lower-level \TeX{} macros.
But if \LaTeX 3 supports what you need, \pkg{expl3} allows you to write code which is safer, more consistent and easier to read.

\textbf{Use \TeX{} macros directly only when you are certain you understand why you need them and what the consequences may be.}
\cs{def}, \cs{gdef}, \cs{xdef}, \cs{let} etc.\ contain \emph{\bfseries no safeguards whatsoever}.
If you say \cs{def}\cs{value}\marg{24}, then \TeX{} will not query your decision, but your document will almost certainly not come out as you wish!

This is because, unlike \cs{newcommand}, \cs{def} does not check whether a command already exists.
If it does not exist, it creates it.
If it does exist, it overwrites it.
As this example shows, the danger here is not just overwriting macros you explicitly use.
\TeX{} and \LaTeX{} rely on all sorts of macros behind the scenes and overwriting those can create just as much mayhem as overwriting the ones you use explicitly.
In fact, these tend to create more chaos because you are that much less likely to realise your mistake.
If you redefine \cs{section} to print ‘Tiddly Winks’, you'll probably figure this out sooner or later.
If you redefine \cs{value} to do this, on the other hand, you may have no clue what is happening, especially if you never use \cs{value} directly and have little idea what it does.

\begin{center}
  \LARGE\textzf{Caveat Emptor \dots}
\end{center}

% END subsec:tex

% END sec:beyond2e

\mode*

\end{document}
