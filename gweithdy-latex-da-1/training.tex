% BEGIN preamble
\svnidlong
{$HeadURL: file:///mnt/between/svn/dysgu/trunk/DoctoralAcademy/gweithdai-latex/gweithdy-latex-da-1/training.tex $}
{$LastChangedBy: cfrees $}
{$LastChangedRevision: 7421 $}
{$LastChangedDate: 2018-03-02 22:05:13 +0000 (Gwe, 02 Maw 2018) $}

\input{config/macros-da.cfg}

\AtBeginDocument
{
  \renewcommand*{\LaTeX}{LaTeX}
  \renewcommand*{\LaTeXe}{LaTeX2e}
  \renewcommand*{\TeX}{TeX}
}

\title{\LaTeX{} I}
\subtitle{Adapted from materials produced by UK-TUG Volunteers}
\date{\cymraeg{Mawrth} 9 March 2018}

% END preamble

\begin{document}

\begin{frame}
  \titlepage
\end{frame}

\maketitle

\tutornote{Time: 10:05}

\tableofcontents

\mode<presentation>%
{
  \begin{frame}{Outline}
	\tableofcontents
  \end{frame}
}

\section{An overview of \LaTeX{}}

\begin{frame}{\LaTeX{} is a powerful system}

  \begin{itemize}
	\item \LaTeX{} can be used for everything from a one page letter to a 1000~page textbook.
	\item Most of \emph{our} examples will be simple.
	\item Complex documents, for example interactive books and these slides, use the same ideas as we're exploring today.
	\item By separating input from output, reusing material becomes much easier.
  \end{itemize}

\end{frame}

\tutornote{Demo documents here}

\begin{frame}{What is \LaTeX{}, and what is \TeX{}?}

  \begin{itemize}
	\item \TeX{} is a typesetting application.
	\item \TeX{} uses \emph{primitives} to determine how to put text on a page.
	\item For most practical purposes, we need a \emph{format} built on top of \TeX{}, for example:
	\begin{itemize}
	  \item Plain \TeX{};
	  \item \LaTeX{};
	  \item Con\TeX{}t.
	\end{itemize}
	\item You can think of \LaTeX{} as an interpreter between you and \TeX{}.
  \end{itemize}

\end{frame}

\begin{frame}{\TeX{} \enquote{engines}}

  \begin{block}{pdf\TeX{}}
	The standard binary program: we'll be using today.
  \end{block}

  \begin{block}{Xe\TeX{}}
	A merger of \TeX{} with modern font technology with support for native Unicode input and bidirectional typesetting.
  \end{block}

  \begin{block}{Lua\TeX{}}
	Also a modern engine: integrates the Lua scripting into \TeX{}.
  \end{block}

\end{frame}

\begin{frame}{What do we need to use \LaTeX{}?}

  \begin{itemize}
	\item A \TeX{} distribution: \TeX{} Live (Windows, Mac, Linux) or MiK\TeX{} (Windows only).
	\item A text editor, \emph{e.g.}~Notepad, TextEdit, Emacs.
	\item A PDF viewer, for example Adobe Reader.
  \end{itemize}

  \pause

  Usually, we use a specialist editor which has
  \begin{itemize}
	\item coloured syntax;
	\item buttons or menus to run \LaTeX{}, \emph{etc.};
	\item an integrated spell checker.
  \end{itemize}
\end{frame}

\tutornote{Time: 10:20}

\section{Getting started}

\begin{frame}{\LaTeX{} is not a word processor}

  \begin{itemize}
	\item \LaTeX{} input is stored as plain text files, usually with the extension \texttt{.tex}.
	\item \LaTeX{} input files contain both the text of the document and \emph{commands}.
	\item Commands start with a backslash, so look like this:
	\cs{example}.
	\item Writing in \LaTeX{} is therefore a bit like \emph{programming} it to produce the document you want.
	\item \emph{Logical} mark up is important in \LaTeX{}: we'll use some almost straight away!
  \end{itemize}

\end{frame}

{
\begin{frame}{Workflow}
  \tikzset{external/export=false}

  \noindent\resizebox{\linewidth}{!}{%
	\begin{tikzpicture}[
	  my flow,
	  ]
	  \path (0,0) node[my block] (edit) {\shortstack{Edit\\\texttt{.tex}\\File}} ;
	  \onslide<2->
		\path (3.5,0) node[my diamond] (errors) {\shortstack{Any\\Errors?}};
		\path [my draw] (edit) -- (errors) node[midway, above=-.25em, anchor=south] {\LaTeX{}};
	  \onslide<3->
	  \path [my draw] (errors) -- ++(0,-2) node[midway,right=-1em] {Yes} -| (edit);
	  \onslide<4->
	  \path (7,0) node[my block] (pdf) {\shortstack{View\\PDF}};
	  \path [my draw] (errors) -- (pdf) node[midway, above=-.25em, anchor=south] {No};
	  \onslide<5->
	  \path (10.5,0) node[my diamond] (change) {\shortstack{Any\\Changes?}};
	  \path [my draw] (pdf) -- (change);
	  \onslide<6->
	  \path [my draw] (change) -- ++(0,-2.5) node[midway,right=-1em] {Yes} -| (edit);
	  \onslide<7->
	  \path (14,0) node[my block] (done) {Done};
	  \path [my draw] (change) -- (done) node[midway, above=-.25em, anchor=south] {No};
	\end{tikzpicture}
  }

\end{frame}
}

\begin{frame}[fragile]{Spacing}

	\begin{itemize}
	  \item \LaTeX{} treats multiple spaces as a single space.
	  \item By default, the space between sentences is slightly larger than the space between words --- can be switched off using \cs{frenchspacing}.
	  \item The tilde (\verb|~|) is used to create a non-breaking space.
	  \item New line characters are treated as a space.
	  \item Paragraph breaks should be indicated by a blank line.
	  \item \LaTeX{} automatically indents paragraphs, except for the first paragraph after a section heading.
	\end{itemize}

\end{frame}

\begin{frame}[fragile]{A simple document}
  \begin{block}{Example}
	\begin{semiverbatim}
	  \alert<2>{\\documentclass}\alert<2,4>{[a4paper,12pt]}\alert<2-3>{\{article\}}
	  \alert<5>{\% A comment in the preamble}
	  \alert<6>{\\begin\{document\}}
	  \alert<7>{\% This is a comment}
	  \alert<8>{This is   a simple document\\footnote\{with a footnote\}.}
	  \alert<8>{This is another sentence in the same paragraph which should}
	  \alert<8>{make the paragraph more than one line long.}

	  \alert<8>{This is a new paragraph.}
	  \alert<6>{\\end\{document\}}
	\end{semiverbatim}
  \end{block}

\end{frame}

\begin{frame}{Overleaf}
  Although you will probably want to try out various editors, viewers etc.\ to find the combination which suits you best, we are going to practise today using the online service,
  \mode<presentation>
  {%
    \href{https://www.overleaf.com/}{Overleaf}.
  }
  \mode<article>
  {%
    Overleaf, at \href{https://www.overleaf.com/}{https://www.overleaf.com/}.
  }

  You will need to start by signing up for an account.

  Please do this now by opening the site in the browser of your choice and using the ‘Sign Up’ button.

  \mode<article>
  {
    If you have no objection, I would appreciate it if you would use \im{\href{https://www.overleaf.com/signup?ref=9f72f613adb5}{https://www.overleaf.com/signup?ref=9f72f613adb5}}
    as it will increase my storage space.
    However, you are under absolutely no obligation to do this and I have no objection at all to your not doing so for whatever reason!

    A short introduction to Overleaf is available at \href{https://www.overleaf.com/tutorial}{https://www.overleaf.com/tutorial} in the form of a tutorial.
  }
\end{frame}

\tutornote{Show demo document in Overleaf: \texttt{example1.tex}.}

\begin{exercise}
  Create the above document in Overleaf.
  While you can use the editor of your choice, start by doing this example in Overleaf.
  Overleaf will automatically save your file with a \texttt{.tex} extension, for example \texttt{main.tex}.
  Overleaf will automatically compile your document and show you a preview on the right.
  To compile your document using:
  \begin{verbatim}
	pdflatex main
  \end{verbatim}
  click the PDF button at the top of the screen.
  Overleaf will ask you whether to open the PDF in your PDF viewer or to save a copy.

  Try experimenting with white space: what do multiple spaces and multiple lines do?
  Also try out using the tilde (\verb"~") for a non-breaking space and \cs{,} for spaces of different widths.

  \LaTeX{} automatically indents new paragraphs: see what the \cs{noindent} macro does to these cases.
\end{exercise}

\tutornote{Finish exercise at 10:45}

\section{Logical structure}

\begin{frame}[fragile]{Logical mark up}

  \LaTeX{} provides us with logical mark up, as well as the ability to directly set the appearance.
  \begin{block}{Logical mark up}
	\alert<2>{\cs{emph}\marg{\meta{text}}}\\
	\marg{\alert<3>{\cs{large}} \meta{text}}
  \end{block}
  \begin{block}{Appearance mark up}
	\alert<2>{\cs{textit}\marg{\meta{text}}}\\
	\marg{\alert<3>{\cs{fontsize}\marg{12 pt}\marg{14 pt}\cs{selectfont}} \meta{text}}
  \end{block}
  Usually, logical mark up is best when it is available.

\end{frame}

\begin{frame}{Title Page}

  First, you need to give the \enquote{meta-data}:
  \begin{itemize}
	\item \cs{title}\marg{\meta{title}}
	\item \cs{author}\marg{\meta{author(s)}}
	\item \cs{date}\marg{\meta{date}} (optional)
  \end{itemize}
  Then use \cs{maketitle} to display the title page.

\end{frame}

\begin{frame}{Sectioning commands}

  \begin{itemize}
	\item
	\alert<4>{\cs{chapter}\alert<2>{\oarg{\meta{short title}}}\alert<3>{\marg{\meta{title}}}}
	\item \cs{section}\alert<2>{\oarg{\meta{short title}}}\alert<3>{\marg{\meta{title}}}
	\item \cs{subsection}\alert<2>{\oarg{\meta{short title}}}\alert<3>{\marg{\meta{title}}}
	\item \cs{subsubsection}\alert<2>{\oarg{\meta{short title}}}\alert<3>{\marg{\meta{title}}}
	\item
	\alert<5>{\cs{paragraph}\alert<2>{\oarg{\meta{short title}}}\alert<3>{\marg{\meta{title}}}}
	\item \cs{subparagraph}\alert<2>{\oarg{\meta{short title}}}\alert<3>{\marg{\meta{title}}}
  \end{itemize}

\end{frame}

\begin{exercise}
  Try producing the following document.
  \verbatiminput{examples/example2}

  Experiment with the logical mark up for appearance, for example \cs{emph}, \cs{large}, \cs{Large} and \cs{Huge}. Compare these with the direct changes brought about by \cs{textit} and \cs{textbf}.

  Try changing the format of text for a longer block by trapping the formatting changes within \cs{begingroup} and \cs{endgroup}, for example
  \begin{verbatim}
	\begingroup
	\large
	\itshape
	Some text

	A second paragraph
	\endgroup
  \end{verbatim}

\end{exercise}

\begin{frame}[fragile,plain]{Lists}

\begin{block}{Order not important}
  \begin{verbatim}
	\begin{itemize}
	  \item This is an unordered list
	\end{itemize}
  \end{verbatim}
\end{block}

\begin{block}{Order important}
  \begin{verbatim}
	\begin{enumerate}
	  \item This one is ordered
	  \item So this will have number 2!
	\end{enumerate}
  \end{verbatim}
\end{block}

\end{frame}

\begin{exercise}
  Make some lists, and nest one list inside another.
  How does the format of the numbers or markers change?
  You can only go to four levels with standard \LaTeX{}, but more than four nestings tends to be a bad sign anyway!
\end{exercise}

\tutornote{Finish exercise at 11:30}

\begin{frame}[fragile]{Quoting things}
  \begin{itemize}
	\item Single quotes: \textasciigrave\meta{short quote}\textquotesingle.
	\item Double quotes: \textasciigrave\textasciigrave\meta{short quote}\textquotesingle\textquotesingle.
	\item Block quotes: \cs{begin}\marg{quotation}\dots\cs{end}\marg{quotation}.
  \end{itemize}
\end{frame}

\begin{exercise}
  Can you produce the following passage?
  \begin{center}
	\fbox{\begin{minipage}{.85\linewidth}
		\normalfont
		The following passage if from Lewis Carroll's \emph{Alice's Adventures in Wonderland}, which is included in \emph{The Complete Works of Lewis Carroll}, published by The Modern Library: Random House (Pp.~75--76):
		\begin{quotation}
		  ‘Come, we shall have some fun now!’ thought Alice.
		  ‘I'm glad they've begun asking riddles --- I believe I can guess that,’ she added aloud.

		  ‘Do you mean that you think you can find out the answer to it?’ said the March Hare.

		  ‘Exactly so,’ said Alice.

		  ‘Then you should say what you mean,’ the March Hare went on.

		  ‘I do,’ Alice hastily replied; ‘at least --- at least I mean what I say --- that's the same thing, you know.’

		  ‘Not the same thing a bit!’ said the Hatter.
		  ‘Why, you might just as well say that ‘I see what I eat’ is the same thing as ‘I eat what I see’!’

		  ‘You might just as well say,’ added the March Hare, ‘that ‘I like what I get’ is the same thing as ‘I get what I like’!’

		  ‘You might just as well say,’ added the Dormouse, which seemed to be talking in its sleep, ‘that ‘I breathe when I sleep’ is the same thing as ‘I sleep when I breathe’!’
		\end{quotation}
	  \end{minipage}}
	\end{center}
  What happens if you replace \env{quotation} with \env{quote} for a block quotation?
\end{exercise}

\tutornote{Finish exercise at 11:50}

\section{Classes}

\begin{frame}{Document classes}

  The \emph{document class} sets up the general layout of the document.
  For example:
  \begin{itemize}
	\item the format of the headings;
	\item if the document should have chapters;
	\item if the title should be on a separate page or above the text on the first page.
  \end{itemize}
  They can also add new control sequences.

  \begin{block}{Usage}
	\cs{documentclass}\oarg{\meta{options}}\marg{\meta{class-name}}
  \end{block}
\end{frame}

\begin{frame}{Base classes}

  \begin{description}
	\item[\cls{article}] for short documents without chapters.
	\item[\cls{report}] for longer documents with chapters, typically single-sided with an abstract.
	\item[\cls{book}] for books, typically double-sided with front matter and back matter.
	\item<2->[\cls{letter}] for correspondence.
	\item<2->[\cls{slides}] for presentations.
  \end{description}

\end{frame}

\begin{frame}{Modern classes}

  \begin{description}
	\item[\cls{KOMA-Script}] \cls{scrartcl}, \cls{scrreprt} and \cls{scrbook} to replace \cls{article}, \cls{report}
	and \cls{book}, respectively.
	\item[\cls{memoir}] replaces \cls{book} and \cls{report}.
	\item[\cls{beamer}] for slides (used to create the material for this workshop).
  \end{description}

\end{frame}

\begin{frame}[fragile,plain]{KOMA-Script Example}

  \verbatiminput{examples/example3}

\end{frame}

\begin{frame}{Documentation}

  \begin{block}{On your computer (after installing a TeX distribution!)}
	The \texttt{texdoc} application will show documentation for material you have installed.
	From the Command Prompt/Terminal
	\begin{center}
	  \texttt{texdoc \meta{name}}
	\end{center}
  \end{block}

  \begin{block}{Online}
	Try CTAN:
	\begin{center}
	  \texttt{http://ctan.org/pkg/\meta{name}}
	\end{center}
	or \texttt{texdoc} online
	\begin{center}
	  \texttt{http://texdoc.net/pkg/\meta{name}}
	\end{center}
  \end{block}

\end{frame}

\begin{exercise}
  Try creating the above document.
  The KOMA-Script classes have various options that affect the document's appearance.
  Try experimenting with some of the following: \texttt{chapterprefix}, \texttt{numbers=enddot}, \texttt{numbers=noenddot}, \texttt{headings=small}, \texttt{headings=normal}, \texttt{headings=big}.
  For example:
  \begin{verbatim}
	\documentclass[chapterprefix]{scrreprt}
  \end{verbatim}

  Also try making simple documents with \textsf{memoir}: see how without any other changes the appearance of the PDF file is altered.

  To add some \enquote{dummy text} to your files as filler, put the line \verb"\usepackage{lipsum}" before \verb"\begin{document}", then put \cs{lipsum} somewhere after \verb"\begin{document}".
  This will create a number of filler paragraphs.
  Use this to see what effect the \texttt{twocolumn} class option has on the layouts you see.

  Use \texttt{http://ctan.org/} or \texttt{http://texdoc.net/} to look up the documentation for the classes we are using.
  Some of these are very long: most of the time you only need a small subset of the commands available!
\end{exercise}

\section{Cross-referencing}

\begin{frame}[fragile]{Cross-referencing}

  \begin{block}{Example input}
	\begin{verbatim}
	  \section{A section}
	  \label{sec:interesting}
	  ...
	  \ref{sec:interesting}
	\end{verbatim}
  \end{block}

  \emph{Two} \LaTeX{} runs are needed to get cross-references right.

\end{frame}

\tutornote{Mention \pkg{fancyref}/\pkg{cleveref}}

\begin{exercise}
  Try producing the following document.
  \verbatiminput{examples/example4}
  \tutornote{Finish exercise at 12:45 and go to lunch.}

  Experiment with cross-references to sections, subsections and items in ordered lists.
  Can you see how it works?
\end{exercise}


\section{More logical structure}\tutornote{Restart at 13:45.}

\begin{frame}[fragile]{Mathematics}

  \begin{itemize}
	\item Mathematical content is marked up in \LaTeX{} in a logical way.
	\item You can use \$ \ldots \$ or \cs{(} \ldots \cs{)} to mark up in-line maths.
	\item For displayed mathematics, use \cs{[} \ldots \cs{]}.
	\item A lot of spacing is automatic in math mode.
	\item Maths is an entire area on its own!
  \end{itemize}

  \begin{block}{Example input}
	\verb"\( y = 2 \sin \theta^2 \)"
  \end{block}

  \begin{block}{Example output}
	\( y = 2 \sin \theta^2 \)
  \end{block}

\end{frame}

\tutornote{Mention AMS material and Vo{\ss}'s \emph{Math Mode}}

\begin{exercise}
  Create some simple mathematical content (for example \verb"y = mx + c") and compare the effect of \cs{(} \ldots \cs{)} with \cs{[} \ldots \cs{]}.
  Can you work out how to get capitalised Greek letters?
  Can you guess why some Greek do not seem to work?

  Subscripts and superscripts in math mode are created using \verb"_" and \verb"^", respectively.
  Try these out, and think about why you might use \cs{textsuperscript} rather than \verb"^" in some cases.

\end{exercise}

\tutornote{Finish exercise at 14:15.}

\section{Packages}

\begin{frame}[fragile]{On packages}

  The \LaTeX{} kernel is rather limited: to get around that we load \emph{packages}:
  \begin{semiverbatim}
	\cs{usepackage}\oarg{\meta{options}}\marg{\meta{package}}
  \end{semiverbatim}
  or
  \begin{semiverbatim}
	\cs{usepackage}\texttt{\{\meta{package1},\meta{package2},\ldots\}}
  \end{semiverbatim}
  We have already seen the \pkg{lipsum} package!

  \vspace{1 em}

  \uncover<2>{%
	Documentation for packages is available in exactly the same way as for classes.%
  }

\end{frame}

\begin{frame}[fragile]{Including external images}

  \begin{itemize}
	\item Load the \pkg{graphicx} package to include graphics.
	\item Use \cs{includegraphics} to actually place the image.
	\item Image formats: \texttt{pdf}, \texttt{png}, \texttt{jpg}.
	\item Images in \texttt{eps} format \enquote{auto-converted} to \texttt{pdf}.
	\item File extension should be omitted.
  \end{itemize}

  \vspace{1 em}
  \uncover<2>{%
	Graphics can also be \enquote{drawn} in \LaTeX{} using the Ti\emph{k}Z package:\\ a course in itself!
  }

\end{frame}

\begin{frame}[fragile]{Options for image inclusion}
  \cs{includegraphics} uses a \textbf{\alert<3,5-6>{key}\alert<5-6>{-}\alert<4,5-6>{value}} interface for configuration.
  This means that although it takes only a single optional argument, this argument can itself specify multiple options.

  \uncover<2->{%
    \begin{semiverbatim}
      \\includegraphics\alert<2>{[}\alert<3,5>{\meta{key for option 1}}\alert<5>{=}\alert<4-5>{\meta{value for option 1}},
      \alert<3,6>{\meta{key for option 2}}\alert<6>{=}\alert<4,6>{\meta{value for option 2}}\alert<2>{]}\{myimage\}
    \end{semiverbatim}
  }

    \uncover<7->{%
   These options can be used to change the display of the image in various ways e.g.~by cropping, scaling or rotating the image.}

\end{frame}


\tutornote{Perhaps include keyval interface for graphics options}

\section{Floating material}

\begin{frame}[fragile]{Floating figures}

  \begin{block}{A floating figure \dots}
	\begin{semiverbatim}
	  \\begin\{figure\}\alert<2>{[htbp]}
	  \alert<3>{\\centering}
	  \\includegraphics\{myimage\}
	  \alert<4>{\\caption\{A Sample Figure\}}
	  \alert<5>{\\label\{fig:sample\}}
	  \\end\{figure\}
	\end{semiverbatim}
  \end{block}

  \begin{block}<6->{\ldots needs a cross-reference}
	\begin{semiverbatim}
	  as is shown in Figure~\alert<6>{\\ref\{fig:sample\}}
	\end{semiverbatim}
  \end{block}

\end{frame}

\begin{exercise}
  Try producing the following document.
  (Use an image application, such as Paint, to produce a simple picture and save it as \texttt{shapes.png}\footnote{If you don't wish to draw anything, try including the image \texttt{example-image-a}.}.)
  \verbatiminput{examples/example5}
  Here are some more class options to try that will affect the list of figures: \texttt{chapteratlists}, \texttt{chapteratlists=0mm}.
  How do the following sets of key-value options for \cs{includegraphics} affect the image included in the document?
  \begin{itemize}
    \item \texttt{width=3cm}
    \item \texttt{height=1cm,width=3cm}
    \item \texttt{height=1cm,width=3cm,keepaspectratio}
    \item \texttt{angle=60,scale=2}
  \end{itemize}
\end{exercise}

\begin{frame}{Tables}

  \begin{itemize}
	\item The floating environment for a table is called \texttt{table}.
	\item However, the content can be anything!
	\item Use the \texttt{tabular} environment to make tables.
	\item Load the \textsf{booktabs} package for rules.
  \end{itemize}

\end{frame}

\begin{frame}[fragile,plain]{Tables}

  \begin{block}{A simple table}
	\begin{semiverbatim}
	  \alert<2>{\\begin\{table\}}
	  \alert<2>{\\centering}
	  \alert<2>{\\caption\{A caption\}}
	  \alert<2>{\\label\{tab:example\}}
	  \alert<3>{\\begin\{tabular\}}\alert<4>{\{lcr\}}
	  \alert<5>{\\toprule}
	  Heading \alert<6>{&} Another one \alert<6>{&} A third \alert<7>{\\\\}
	  \alert<5>{\\midrule}
	  a \alert<6>{&} b \alert<6>{&} c \alert<7>{\\\\}
	  d \alert<6>{&} e \alert<6>{&} f \alert<7>{\\\\}
	  \alert<8>{\\multicolumn\{3\}\{c\}\{Wide text\}} \alert<7>{\\\\}
	  \alert<5>{\\bottomrule}
	  \alert<3>{\\end\{tabular\}}
	  \alert<2>{\\end\{table\}}
	\end{semiverbatim}
  \end{block}

\end{frame}

\begin{exercise}

  Use the simple table example to start experimenting with tables.
  Try out different alignments using the \texttt{l}, \texttt{c} and \texttt{r} column types.
  What happens if you have too few items in a table row?
  How about too many?
  Experiment with the \cs{multicolumn} command to span across columns.

\end{exercise}

\tutornote{Finish exercise at 15:15.}

\section{Multilingual typesetting}

\begin{frame}{Beyond US English}
  \TeX{} was originally designed for US English.

  By default, \LaTeX{} assumes your document uses US English.

  It is important to tell \LaTeX{} if this is not the case!

  \begin{itemize}
    \item \pkg{babel} configures your document for one or more languages.
    \item \pkg{inputenc} supports different \emph{input} encodings.
    \begin{itemize}
      \item \cs{usepackage}\oarg{utf8}\marg{inputenc} enables you to input (many) accented characters directly.
    \end{itemize}
    \item \pkg{fontenc} supports different \emph{font} (output) encodings.
    \begin{itemize}
      \item \cs{usepackage}\oarg{T1}\marg{fontenc} supports most Western European languages.
    \end{itemize}
  \end{itemize}

\end{frame}
\begin{frame}[fragile]

  \begin{block}{Preamble}
    \begin{semiverbatim}
      \cs{documentclass}\alert<1>{\oarg{\meta{language options}}}\marg{\meta{document class}}
      \alert<1>{\cs{usepackage}\marg{babel}}
      \alert<2>{\cs{usepackage}\oarg{utf8}\marg{inputenc}}
      \alert<3>{\cs{usepackage}\oarg{T1}\marg{fontenc}}\% usually what you want
    \end{semiverbatim}
  \end{block}

\end{frame}
\begin{frame}[fragile]

  \begin{block}{Syntax}
    \begin{semiverbatim}
      \% switch to \meta{other language} until further notice
      \alert<1>{\cs{selectlanguage}\marg{\meta{other language}}}
      \% typeset short \meta{text} in \meta{other language}
      \alert<2>{\cs{foreignlanguage}\marg{\meta{other language}}\marg{\meta{text}}}
      \% temporary switch to \meta{other language}
      \alert<3>{\\begin\marg{otherlanguage}\marg{\meta{other language}}
        \meta{text}
      \\end\marg{otherlanguage}}
    \end{semiverbatim}
  \end{block}

\end{frame}

\begin{frame}[fragile]

  \begin{block}{UK English}
    \begin{semiverbatim}
      \cs{documentclass}\oarg{\alert{british}}\marg{article}
      \cs{usepackage}\marg{babel}
      \% recommended even for English
      \cs{usepackage}\oarg{utf8}\marg{inputenc}
      \cs{usepackage}\oarg{T1}\marg{fontenc}
    \end{semiverbatim}
  \end{block}

\end{frame}
\begin{frame}[fragile]
  \begin{block}{German (Default) and US English}
    \begin{semiverbatim}
      \cs{documentclass}\oarg{\alert<1,4>{american}\alert<1>{,}\alert<1,3>{ngerman}}\marg{article}
      \cs{usepackage}\marg{babel}
      \cs{usepackage}\oarg{utf8}\marg{inputenc}
      \cs{usepackage}\oarg{T1}\marg{fontenc}
    \end{semiverbatim}
  \end{block}
  \onslide<2->
  \begin{block}{US English (Default) and German}
    \begin{semiverbatim}
      \cs{documentclass}\oarg{\alert<2,4>{ngerman}\alert<2>{,}\alert<2,3>{american}}\marg{article}
      \cs{usepackage}\marg{babel}
      \cs{usepackage}\oarg{utf8}\marg{inputenc}
      \cs{usepackage}\oarg{T1}\marg{fontenc}
    \end{semiverbatim}
  \end{block}
\end{frame}

\begin{frame}[fragile]

  \begin{block}{UK English (Default) and Welsh}
    \begin{semiverbatim}
        \cs{documentclass}\oarg{\alert<1>{welsh,british}}\marg{article}
        \cs{usepackage}\marg{babel}
        \cs{usepackage}\oarg{utf8}\marg{inputenc}
        \% uncomment for older LaTeX for Ŵ ŵ Ŷ ŷ
        \% \alert<3>{\cs{input}\marg{ix-utf8enc.dfu}}
        \cs{usepackage}\oarg{T1}\marg{fontenc}
        \alert<2>{\cs{providecommand}\marg{\cs{cymraeg}}\oarg{1}\marg{\%
          \cs{foreignlanguage}\marg{welsh}\marg{#1}}}
    \end{semiverbatim}
  \end{block}
\end{frame}

\begin{exercise}
    Experiment with the effects of different languages.
    What effect does changing the language have on the output of \cs{today}?

    \textbf{Note that you need to delete generated files after changing language options.}
    To do this on OverLeaf, choose ‘compile from scratch’.
\end{exercise}

\tutornote{Finish exercise at 15:45.}

\section{Bibliographies}

\begin{frame}{Creating a bibliography}

  There are 3 common ways to create a bibliography in \LaTeX{}.
  \begin{itemize}
	\item Use the \env{thebibliography} environment.
	\item Use a database of bibliographical entries and \BibTeX{}.
	\item Use a database of bibliographical entries, \pkg{biblatex} and Biber.
  \end{itemize}

\end{frame}

\begin{frame}[fragile]{Creating a bibliography}{The \LaTeX{} basics}
  To create the bibliography, use \env{thebibliography}:
  \begin{semiverbatim}
	\\begin\{thebibliography\}\marg{\meta{sample label}}
	...
	\\end\{thebibliography\}
  \end{semiverbatim}
  \pause
  Each entry looks like this:
  \begin{semiverbatim}
	\cs{bibitem}\oarg{\meta{label}}\marg{\meta{key}} \meta{entry text}
  \end{semiverbatim}
  Citations in the text then use the defined \texttt{key}:
  \begin{semiverbatim}
	  Some text \\cite\marg{\meta{key}}.
  \end{semiverbatim}
\end{frame}

\begin{frame}[fragile]{\texttt{thebibliography}}
\begin{example}
\begin{semiverbatim}
\alert<1>{\\begin\{thebibliography\}\{99\}}
	\alert<2>{\\bibitem\{lamport\}}\alert<3>{
	  Leslie Lamport, \\textit\{\\LaTeX\{\}: A Document
	  Preparation System\}. Second Edition. Addison-Wesley:
	  Reading, MA. 1994.}
\alert<1>{\\end\{thebibliography\}}
\end{semiverbatim}
\end{example}
\end{frame}

\begin{frame}[fragile]{\texttt{thebibliography}}{A journal article}

\begin{example}
\begin{semiverbatim}
\\begin\{thebibliography\}\{99\}
	\alert<1>{\\bibitem\{smith05\}}\alert<2>{
	  John Smith, Jr, Jane Lucy Doe, Andrew N.\\ Other and
	  Jo de Vere, `An imaginary article'. In
	  \\textit\{Journal of the Imaginary Society\}
	  XIV(4):45--67. 2005.}
\\end\{thebibliography\}
\end{semiverbatim}
\end{example}

\end{frame}

\begin{frame}[fragile]{Citing the entries}
  \begin{example}
	\begin{semiverbatim}
	  For additional information about \\LaTeX\{\} and \\TeX\{\} see
	  \\cite\{lamport\} and \\cite\{smith05\}.
	\end{semiverbatim}
  \end{example}
\end{frame}

\section{Long documents}

\begin{frame}{Working with long documents}

  \begin{itemize}
	\item Long documents are best split into parts.
	\item \cs{input} places the material loaded \enquote{here}.
	\item \cs{include} is used for separate chapters:\\ it always starts a new page.
	\item Using \cs{include} allows you to \cs{includeonly}\\
	selected chapters.
	\item Use \cs{includeonly} in the preamble.
  \end{itemize}

\end{frame}

\section{Page layout}

\begin{frame}[fragile]{Changing the page layout}

  By default, the standard classes use US letter sized paper.

  For consistent A4 layout with standard classes:
  \begin{semiverbatim}
    \cs{documentclass}\oarg{\alert{a4paper}}\marg{article}
    \alert{\cs{usepackage}\marg{geometry}}
  \end{semiverbatim}

  \begin{itemize}
    \item Use \cs{geometry} with \cls{article}, \cls{report}, \cls{book} and \cls{letter}.
    \item \pkg{geometry} is usually best avoided when using classes such as \cls{memoir} or KOMA-Script.
    \item Use \cs{geometry}\marg{\meta{options}} in the preamble to customise the layout.
  \end{itemize}

\end{frame}

\begin{exercise}

  Create a master file \verbatiminput{examples/example7.tex} along with the three chapters, each of which can be as simple as
  \begin{verbatim}
	\chapter{A demo}
	\lipsum
  \end{verbatim}
  Experiment with this basic structure, and using \cs{includeonly} to use only some of the files.

\end{exercise}

\begin{exercise}

  Working with the document you created in the previous exercise, try adding
  \begin{verbatim}
    \geometry{top=10mm,bottom=50mm,left=40mm,right=5mm,showframe}
  \end{verbatim}
  before \verb|\begin{document}|.
  What happens if \verb|a4paper| is changed to \verb|a5paper|?

\end{exercise}

\tutornote{Finish exercise at 16:30.}

\section{Solving problems}

\begin{frame}{Problem solving strategies}
	\begin{itemize}
		\item Revert to the last known-good state of your document.
		  Then reintroduce the changes one-by-one.

		  (Only works if you keep backups, but you do that anyway, right?)
		\item Comment out code (using \texttt{\%}) until you eliminate the problem.
		  Then uncomment gradually.
		\item Take a break!
		  Staring at non-working code blinds you to the obvious.
		\item Make the smallest document you can which demonstrates the problem. If that doesn't help, ask.
	\end{itemize}
\end{frame}

\begin{frame}{Describe the symptoms}
	Be clear about the problem.

	If you are asking online, people only know what you tell them!

	There are different kinds of problems.
	\begin{itemize}
		\item Compilation failure.
		  (No PDF is produced.)

		  What is the error message?

		  What does the \texttt{.log} say?
		\item PDF compiles but with errors.

		  Which errors?
		\item No errors but the output is ‘wrong’.

		  What exactly is wrong with it?
	\end{itemize}

\end{frame}

\begin{frame}{Solution strategies}
	\begin{itemize}
		\item Delete ‘generated’ files e.g.\ \texttt{.aux}.
		\item Check that every \cs{begin}\marg{\meta{environment}} matches an \cs{end}\marg{\meta{environment}}.
		\item Check that the \textt{\{}s match the \texttt{\}}s.
		\item Make sure you don't start writing before \cs{begin}\marg{document}.
	\end{itemize}

\end{frame}

\begin{exercise}
What's wrong?

\verbatiminput{examples/example8.tex}

\verbatiminput{examples/example9.tex}

Examples are from Werner's answer at \url{http://tex.stackexchange.com/a/33472/}.

\end{exercise}

\tutornote{Finish exercise at 16:50.}


\tutornote{Any remaining time for questions/further experimentation.}

\tutornote{Time allowing, simple \textsf{hyperref} demo is good here}

\mode
<all>
\furtherinfo[%
  Appendix \ref{handouts:sec:tex-distros}: \TeX{} Distributions.
][and on OverLeaf at\bigskip\par\url{overleaf.com/read/yhcyjnnfmbhk}]
\mode*

\clearpage
\appendix

\section<1-| beamer:0>{Installing a \TeX{} Distribution}\label{sec:tex-distros}
% BEGIN sec:tex-distros

For learning \LaTeXe{}, an online compiler works well: you don't have to worry about installation, don't need to find disk space and can start experimenting as soon as you've registered an account.

For more serious usage, however, a local installation of \TeX{} is recommended.
If you will be handling confidential material, this is \textbf{essential}: \textbf{do not upload private or confidential material to any online service}.

Even if you do not handle confidential material and are unconcerned about the security and privacy of your materials, a local installation offers significant advantages.
Local compilation is more flexible, predictable and efficient than an online service.
Overleaf is \emph{slloooooooooowww}; local compilation is, in contrast, comparatively fast, even on moderately-powered moderately-ancient hardware.

Crucially, a local compilation allows you to avoid updating close to a crucial deadline: \textbf{never update your installation when close to completing a critical project}\footnote{%
  The exception to this is that it is usually fine to install an entirely new version of your \TeX{} distribution, if you are able to do so alongside your existing one --- in that case, you aren't losing your existing installation and can revert if necessary.}.

\subsection<1-| beamer:0>{Needful Things}\label{subsec:needful}
% BEGIN subsec:needful

Recall that to install a working \TeX{} configuration, you need three things:
\begin{enumerate}
  \item a \TeX{} distribution;
  \item an editor;
  \item a document viewer (usually PDF).
\end{enumerate}
The critical choice here is the first: it is possible to install as many editors and viewers as your disk space allows, experimenting until you find a suitable combination.
But it is generally unwise to install multiple \TeX{} distributions, especially if you have relatively little experience with the system.
Mac OS X and Windows users will find that any distribution comes with an editor.
However, users are encouraged to try different editors.
Users of other operating systems will need to install at least one editor separately.
Except for Windows, most systems probably include a suitable document viewer already.
Windows users should note that use of Adobe Reader is problematic, so an alternative viewer is recommended.
\bigskip

The remainder of this appendix concentrates on the choice of \TeX{} distribution.
See \cref{fig:choice-distro} for an overview.

\begin{sidewaysfigure}
  \begin{forest}
    flow tree,
    tikz+={
      \draw [decorate, decoration={brace, amplitude=10pt}, ultra thick, bcol1] ([yshift=-5pt]!LPuu.east |- current bounding box.south) coordinate (a) -- (current bounding box.west |- a) node [midway, font=\bfseries\sffamily, below, anchor=north, yshift=-10pt, bcol1, label=below:\url{www.tug.org/texlive/}] {\TeX{} Live};
    }
    [Which operating system?, to choose
      [Mac OS X, my choice
        [Mac\TeX{} (\TeX{} Live for OS X), selection, tier=distro
          [www.tug.org/mactex/, url]
        ]
      ]
      [Unix-ish, my choice
        [\TeX{} Live, selection, tier=distro
          [What matters most?, to choose, for children={my choice, align={@{}b@{}}}
            [Easy install\\System integration
              [{Distro's \texttt{texlive} packages\\(Install with e.g.~\texttt{apt}, \texttt{dnf}, \texttt{pacman})}, selection, tier=unx]
            ]
            [{Multi-version support (e.g.~2017 + 2018)\\Non-root install \& maintenance possible\\Latest packages, bug fixes \& updates}
              [Upstream's installer\\\& dummy \texttt{texlive} package(s), selection, tier=unx
                [www.tug.org/texlive/quickinstall.html, url]
              ]
            ]
          ]
        ]
      ]
      [Windows, my choice
        [Cross-platform solution desirable?, to choose
          [Cross-platform, my choice
            [\TeX{} Live, selection, tier=win distro]
          ]
          [Windows only, my choice
            [On-the-fly package installation needed?, to choose
              [On-the-fly unneeded, my choice
                [\TeX{} Live, selection, tier=win distro
                  [www.tug.org/texlive/windows.html, url, before drawing tree={x/.process={OOw2+d{!u.x}{!uP.x}{(#1+#2)/2}  }}]
                ]
              ]
              [On-the-fly, my choice
                [Mik\TeX, selection, tier=win distro
                  [miktex.org, url]
                ]
              ]
            ]
          ]
        ]
      ]
    ]
  \end{forest}
  \caption{Selection of \TeX{} distribution and installation method}\label{fig:choice-distro}
\end{sidewaysfigure}

% END subsec:needful

\subsection<1-| beamer:0>{Windows}\label{subsec:win}
% BEGIN subsec:win

If you are a Windows user, you have two main choices:
\begin{enumerate}
  \item Mik\TeX{};
  \item \TeX{} Live.
\end{enumerate}
The first is specifically designed for Windows, allows on-the-fly installation of packages during compilation (so you can install a minimal set of software and pull down additions only if you need them) and is well-maintained, widely used and well-supported.
The second is cross-platform, well-maintained, very widely used and very well-supported.
Mik\TeX{} has one or two quirks which \TeX{} Live does not.
In particular, Mik\TeX{} users sometimes need to update font map files when these are not auto-regenerated upon the installation of new font packages. 
In addition, Mik\TeX{} includes some packages which are excluded from \TeX{} Live.
To use these packages on \TeX{} Live, you would need to install them manually.
However, a set of excluded font packages is supported by an installation script, so this generally applies to non-font packages (which are less tricky to install manually than font packages).
I have never used Mik\TeX{}, so I cannot really recommend for or against it.
Online compilation services are pretty much guaranteed to be using \TeX{} Live.

% END subsec:win

\subsection<1-| beamer:0>{Not Windows}\label{subsec:not-win}
% BEGIN subsec:not-win

Use \TeX{} Live.

% END subsec:not-win

\subsection<1-| beamer:0>{Installing \TeX{} Live}\label{subsec:tl}
% BEGIN subsec:tl

Pre-compiled binaries are available for most operating systems.
Users of very unusual systems might need to compile from source, but if you are using such a system, you are probably used to this and can find instructions at \url{www.tug.org/texlive/build.html}.

\subsubsection<1-| beamer:0>{Mac OS X}\label{subsubsec:osx}
% BEGIN subsubsec:osx

If you are using Mac OS X, you should install the latest version of Mac\TeX{} available for your system.
Do NOT install using home brew or any similar ports system.

% END subsubsec:oxs

\subsubsection<1-| beamer:0>{Unix-ish (GNU/Linux, FreeBSD etc.)}\label{subsubsec:unx}
% BEGIN subsubsec:unx

[I know that Mac OS X is Unix-ish, too, but it is nonetheless excluded from the set of ‘Unix-ish’ operating systems for current purposes.]

If you are using GNU/Linux, FreeBSD or a similar Unix-ish system, you may choose to install either from upstream or using your operating system distribution's package manager (e.g.~\texttt{pacman}, \texttt{apt}, \texttt{dnf}, \texttt{yum}).

I generally recommend installing from upstream.
However, there is no doubt that doing so adds an additional layer of complexity to managing your system.
\textbf{If you install from upstream, you need to install a ‘dummy’ package to satisfy dependencies when installing an editor etc.\ through your system's package manager.}
This is the most significant disadvantage of using the upstream installer.

The advantage is that it makes managing \TeX{} Live extremely straightforward.
In general, distro-repackaged versions of \TeX{} Live are fragmented, outdated (often by years) and trickier to navigate.
Repackaged versions typically include changes which alter the location of configuration files and disable most of the functionality offered by \TeX{} Live's own package manager. 
(They have to disable most of the package manager's functionality to prevent it from conflicting with the system's primary package manager.
Only one package manager can sensibly manage the same directories.)

That said, \TeX{} Live relies on system-installed libraries.
If your system includes different versions and this breaks \TeX{} Live's binaries, that is now your problem.
If you are using distro-repackaged \TeX{} Live, it is the task of the people doing the packaging to adjust things so that this doesn't happen.
So, if it breaks, it is somebody else's problem. 
Whether that helps or not depends on the packagers \dots. 
Fortunately, this happens rather rarely regardless of installation method, tends not to break anything critical and is generally straightforward to resolve.
The main downside of upstream's installer remains, therefore, the need for a dummy package.

% END subsubsec:unx

\subsubsection<1-| beamer:0>{Pre-Installation}\label{subsubsec:pre}
% BEGIN subsubsec:pre

Before installing, you should read the installation guide for your system:

\begin{tabular}{@{}ll@{}}
  Unix-ish & \url{www.tug.org/texlive/quickinstall.html}\\
  Windows & \url{www.tug.org/texlive/windows.html}\\
\end{tabular}

\noindent Complete documentation is available at \url{www.tug.org/texlive/doc.html}.

On Unix-ish systems, I recommend not installing with root privileges.
Doing so is not necessary and not recommended by upstream.
\url{tex.stackexchange.com/a/187379/} explains how to create and use a dedicated user account for managing \TeX{} Live on a GNU/Linux system.

Whether you install as root or not, do remember to install a dummy package.
If you do not do this, you will end up with multiple installations which will not get along at all well together, typically causing mysterious and confusing errors.
(It is, however, fine to have multiple editions of \TeX{} Live itself installed e.g.~2017, 2018 and 2011.)

\url{tex.stackexchange.com/q/1092} explains ways to install from upstream on Debian-based systems and is easily adapted to other cases. 
This includes information about dummy packages to keep your distro's package manager happy, scripts to automate auxiliary tasks and discussion of various things you might want to consider.
Think carefully before relying on a script available here: if you do not understand what it does or why, you should avoid using it.
Note that recommendations frequently change as new software becomes available, so an unmaintained script may be worse than useless.
Instructions which are not updated are also problematic, but you are more likely to discover the problem and less likely to be faced with a completely mysterious mess.
Moreover, instructions on this page are well-maintained by the community.
Scripts linked from here may be rather less so.
Read through the answers here before proceeding: different solutions are best suited to different situations.
The highest-rated or accepted answer may not be the best for your situation.
(If you are using Fedora, for example, relying on a automagical script designed for Ubuntu is likely to end in tears, but step-by-step instructions may well apply with relatively minimal modifications.)

% END subsubsec:pre

\subsubsection<1-| beamer:0>{Installation}\label{subsubsec:inst}
% BEGIN subsubsec:inst

Assuming you are installing from upstream or are a Windows or OS X user, download as follows:

\begin{tabular}{@{}ll@{}}
  Unix-ish  & \url{mirror.ctan.org/systems/texlive/tlnet/install-tl-unx.tar.gz}\\
  Mac OS X &  \url{www.tug.org/mactex/}\\
  Windows & \url{mirror.ctan.org/systems/texlive/tlnet/install-tl-windows.exe}\\
  & \emph{or} \url{mirror.ctan.org/systems/texlive/tlnet/install-tl.zip}\\
\end{tabular}

\noindent Aside from Mac\TeX{}, these links will download the net install which will download the software from the internet during installation.
Unless you are a Mac OS X user, you should use the net installer unless you have very good reason to do otherwise. 

Installing the distribution is just a matter of starting the installer and following the instructions.
On Windows, the net-installer uses a graphical mode by default; on all other systems, it defaults to the non-graphical mode.
Mac\TeX{} provides a standard OS X-style graphical installer.
You will be asked to make a number of choices.
The defaults are generally good, but make sure you specify A4 paper as default.

Unless you are very short of disk space, install the ‘full’ scheme.
Trying to install ‘only want you need’ is a recipe for frustration and a fruitless waste of effort better spent reading your tea leaves. 
(Staring at tea leaves is equally pointless, but provides greater opportunities for creative self-expression.)

% END subsubsec:inst

\subsubsection<1-| beamer:0>{Post-Installation}\label{subsubsec:post}
% BEGIN subsubsec:post

Post-installation, see \url{www.tug.org/texlive/pkginstall.html} for ways to update the software.
However, note that this is not particularly recommended unless you have a problem and want the latest fixes or wish to ensure you have all the newest bits and bobs.
In particular, \textbf{do not update close to a critical deadline}.
If you must update something close to a deadline (e.g.~you need a bug fix), try downloading the fixed package manually and see if just updating the problematic package is enough to solve the problem.
\emph{After} the deadline, you can delete your manual download and update \TeX{} Live properly.

\mode<article>
{%
  \subsubsection*{Additional Font Packages}\label{subsubsec:fonts}
}
% BEGIN subsubsec:fonts (*)

Some additional font packages may be installed using a script provided by the \TeX{} User Group.
Download the script from \url{tug.org/fonts/getnonfreefonts/}.
\textbf{Do NOT use the \texttt{--user} option!}
If you do this by accident, it will initially work fine, but will cause you no end of trouble later on.
That is, when you've long forgotten about what you did, things will suddenly go pear-shaped and you will have no idea why.
Instead, follow the directions for installing the fonts system-wide. 
\begin{verbatim}
  getnonfreefonts --sys [other arguments]
\end{verbatim}

% END subsubsec:fonts (*)
  
% END subsubsec:post

% END sec:tex-distros


\end{document}
