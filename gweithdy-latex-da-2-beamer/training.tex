% BEGIN preamble
\svnidlong
{$HeadURL: file:///mnt/between/svn/dysgu/trunk/DoctoralAcademy/gweithdai-latex/gweithdy-latex-da-2-beamer/training.tex $}
{$LastChangedBy: cfrees $}
{$LastChangedRevision: 7490 $}
{$LastChangedDate: 2018-04-24 14:48:48 +0100 (Maw, 24 Ebr 2018) $}

% !TEX TS-program = pdflatex
% !TEX encoding = UTF-8 Unicode
\usepackage{svn-prov}
\ProvidesFileSVN{$Id: macros-da.cfg 6409 2017-05-15 23:22:56Z cfrees $}[\filebase \revinfo]
\usepackage{svn-multi}
	\svnidlong
		{$HeadURL: file:///mnt/between/svn/dysgu/trunk/DoctoralAcademy/config/macros-da.cfg $}
		{$LastChangedBy: cfrees $}
		{$LastChangedRevision: 6409 $}
		{$LastChangedDate: 2017-05-16 00:22:56 +0100 (Maw, 16 Mai 2017) $}
% best not to mix \svnid and \svnidlong (or to always typeset time-zone with time? see documentation)
%	\svnid{$Id: macros-da.cfg 6409 2017-05-15 23:22:56Z cfrees $}
	\svnRegisterAuthor{cfrees}{Clea F. Rees}

% ----------------------------------------------------------------------
% New commands for these documents
% ----------------------------------------------------------------------
\newcommand*{\BibTeX}{BibTeX}
\newcommand*{\cls}[1]{\textsf{#1}}
\newcommand*{\pkg}[1]{\textsf{#1}}
\newcommand*{\cs}[1]{\texttt{\textbackslash#1}}
\newcommand*{\env}[1]{\texttt{#1}}
\newcommand*{\marg}[1]{\texttt{\{#1\}}}
\newcommand*{\oarg}[1]{\texttt{[#1]}}
\newcommand*{\meta}[1]{\ensuremath{\langle}\emph{#1}\ensuremath{\rangle}}
\newcommand*{\filename}[1]{\texttt{#1}}
\newcommand*{\narg}[1][1]{\texttt{\##1}}
\newcommand*{\Tikz}{Ti\textit{k}Z}
\newcommand*{\pgf}{\textrm{\textsc{pgf}}}
\newcommand*{\tikzpgf}{\Tikz/\pgf}
\newcommand*{\pgfplots}{\textrm{\textsc{pgfplots}}}
\newcommand*{\pgfplotstable}{\textrm{\textsc{PgfplotsTable}}}


\endinput


\title{\LaTeX{} II: Beamer}
\subtitle{Based in part on materials produced by UK-TUG Volunteers}
\date{\cymraeg{Ebrill} 26 April 2018}

\mode<presentation>
{
  \newcolumntype{q}{l@{\hskip 2.5pt}}
}
\mode<article>
{
  \newcolumntype{q}{l}
}

% END preamble

\begin{document}

\begin{frame}
  \titlepage
\end{frame}

\maketitle

\tutornote{Time: 10:00}

\tableofcontents

\mode<presentation>%
{
  \begin{frame}{Outline}
	\tableofcontents
  \end{frame}
}

% Topics covered will include:
%
% Q presentations --- Beamer

\section{Introduction}

\begin{frame}{\LaTeX{}: Key features}

  \begin{itemize}
	\item \TeX{} is a typesetting application.
	\item \TeX{} uses \emph{primitives} to determine how to put text on a page.
	\item \LaTeX{} is  a \emph{format} built on top of \TeX{}.
	\item \LaTeX{} can be used for everything from a one page letter to a 1000~page textbook.
	\item By separating input from output, reusing material becomes much easier.
	\item By separating \emph{content} from \emph{formatting}, we can create flexible, consistent documents which are easier to maintain and modify.
	\item We are using the \emph{engine} pdf\TeX{}.
	Xe\TeX{} and Lua\TeX{} are modern alternatives.
  \end{itemize}

\end{frame}

\mode<all>
\overleaf
\mode*

\section{Beamer Presentations}\tutornote{10:15}

\begin{frame}{\LaTeX{} presentations}

  Various packages support the creation of presentations e.g.~slides, posters etc.

  \pkg{beamer} is probably the most popular.

\end{frame}

\begin{frame}{Beamer: An overview}

  \begin{block}{Selected features}
	\pkg{beamer} provides a dedicated class, \cls{beamer}.
	\begin{itemize}
	  \item One file to produce presentation, handouts, notes and article.
	  \item ‘Overlay’ specifications for incremental uncovering of content.
	  \item Mark up for organising, structuring and formatting the presentation.
	  \item By changing the theme, you can change the style of your whole presentation.
	  \item Highly customisable using \LaTeX{} macros.
	\end{itemize}
  \end{block}

\end{frame}

\begin{frame}[fragile,plain]
	\begin{semiverbatim}
		\\documentclass\{\alert<2>{beamer}\}
		\alert<3>{\\mode<presentation>}
		\\title\{Great Stuff\}
		\\author\{Me\}
		\\date\{11 June, 2015\}
		\\begin\{document\}
		\alert<4>{\\begin\{frame\}}
		  \alert<5>{\\titlepage}
		\alert<4>{\\end\{frame\}}
		\\begin\{frame\}
		  \alert<7>{\\frametitle\{Outline\}}
		  \alert<6>{\\tableofcontents}
		\\end\{frame\}
		\\section\{First Section\}
		\\begin\{frame\}\alert<7>{\{Frame Title\}}
		  Content of frame.
		  \alert<8>{\\pause} More content.
		\\end\{frame\}
		\\end\{document\}
	\end{semiverbatim}
\end{frame}

\begin{exercise}
  Create a simple presentation using \pkg{beamer} with a title page, table of contents and one or two sections.

  A single \emph{frame} can contain multiple \emph{slides}.

  When you use \cs{pause}, an additional slide is added to the current frame.

  Try adding the following list to a frame of your presentation:
  \begin{verbatim}
	\begin{itemize}[<+-| alert@+>]
		\item first
		\item second
		\item third
	\end{itemize}
  \end{verbatim}

  What does \cs{alt}\meta{2}\marg{text}\marg{different text} do?

  By adding \cs{usetheme}\marg{\meta{theme name}} to your preamble, you can change how your presentation looks.
  Try experimenting with some of these themes: \textt{EastLansing}, \textt{Berkeley} and \texttt{Copenhagen}.

  Adding \cs{usecolortheme}\marg{\meta{theme name}} allows you to customise the colours.
  Experiment with some of these: \texttt{albatross}, \texttt{crane} and \texttt{dove}.

\end{exercise}

\section{Beamer Themes}\tutornote{11:00}

\begin{frame}{Beamer Themes}
  Beamer customisation is based on the concept of \alert<1>{\textbf{themes}}.

  There are 5 kinds of theme:
  \begin{description}
    \item<1-| alert@2>[presentation] affects all aspects of the presentation's appearance;
    \item<1-| alert@3>[color] determines the colours used in the presentation;
    \item<1-| alert@4>[font] determines the fonts used in the presentation;
    \item<1-| alert@5>[inner] affects the appearance of non-colour, non-font aspects of the ‘inside’ of frames e.g.~format of labels in list environments, contents, theorems etc.;
    \item<1-| alert@6>[outer] affects the appearance of non-colour, non-font aspects of the ‘outside’ of frames e.g.~headlines, footlines, side bars, logos, frame titles etc.
  \end{description}

\end{frame}

\begin{frame}{Specifying Themes}
  The \alert<1>{order} in which themes are specified matters.

  If a presentation theme is almost what you want, but you don't like the colours, for example, specify the presentation theme \textbf{before} the colour theme, so that the latter overrides the former's colours.
  \bigskip
  \onslide<2->

  \centering
  \makebox[0pt]{%
    \begin{tabular}{ql}
      \toprule
      \textbf{Type} & \textbf{Command}\\
      \midrule
      \alert<2>{presentation} & \alert<2>{\cs{usetheme}\oarg{\meta{options}}\marg{\meta{theme name list}}}\\
      \alert<3>{color} & \alert<3>{\cs{usecolortheme}\oarg{\meta{options}}\marg{\meta{color theme name list}}}\\
      \alert<4>{font} & \alert<4>{\cs{usefonttheme}\oarg{\meta{options}}\marg{\meta{font theme name}}}\\
      \alert<5>{inner} & \alert<5>{\cs{useinnertheme}\oarg{\meta{options}}\marg{\meta{inner theme name}}}\\
      \alert<6>{outer} & \alert<6>{\cs{useoutertheme}\oarg{\meta{options}}\marg{\meta{outer theme name}}}\\
      \bottomrule
    \end{tabular}
  }

\end{frame}

\begin{exercise}
  Use the list of built-in themes included in appendix \ref{sec:builtin-themes} to experiment with the effects of installing different kinds and combinations of themes.
\end{exercise}

\section{Beamer Templates}\label{sec:templates}\tutornote{11:20}

% BEGIN sec:templates

\begin{frame}[fragile]{Beamer Templates}
  Fine-grained customisation can be achieved by directly modifying Beamer \alert<1>{\textbf{templates}}.

  \onslide<2->
  Here's an example from the manual (168),
  \begin{semiverbatim}
    \alert<6,7>{\\setbeamercolor}\alert<6,8,12>{\{frametitle\}}\alert<6>{\{fg=red\}}
    \alert<9,10>{\\setbeamerfont}\alert<9,11,12>{\{frametitle\}}\alert<9>{\{series=\bfseries\}}
    \alert<2>{\\setbeamertemplate}\alert<3,12>{\{frametitle\}}
    \alert<4>{\{
      \\begin\{centering\}
        \alert<5>{\\insertframetitle}\\par
      \\end\{centering\}
    \}}
  \end{semiverbatim}
\end{frame}

\begin{frame}{Inheritance}
  Templates are organised \alert<2>{\textbf{hierarchically}}.

  By default, child templates inherit attributes from their parent.

  \begin{block}<2->{Templates: Sample Inheritance}
    \centering
    \begin{forest}
      for tree={
        font=\ttfamily,
      },
      where level=1{
        forked edge,
        for tree={
          grow'=0,
          folder,
        },
      }{}
      [items
        [itemize items
          [itemize item]
          [itemize subitem]
          [itemize subsubitem]
        ]
        [, phantom, calign with current]
        [enumerate items
          [enumerate item]
          [enumerate subitem]
          [enumerate subsubitem]
        ]
      ]
    \end{forest}
  \end{block}

\end{frame}

\begin{exercise}
  Try modifying the appearance of frames' titles following the pattern of the example above.

  What happens to nested \verb|itemize| and \verb|enumerate| environments if you add
  \begin{verbatim}
    \setbeamertemplate{items}[ball]
  \end{verbatim}
  to your preamble?
  Generally, you will want to make changes to templates \textbf{after} loading any themes you wish to use, since the latter may otherwise override the former.
\end{exercise}

% END sec:templates

\section{Beamer Colours}\label{sec:colours}\tutornote{11:40}

% BEGIN sec:colours

\begin{frame}{Beamer Colours}
  Beamer uses \pkg{xcolor} for colour.

  However, it builds a specialised framework on top of the facilities \pkg{xcolor} provides.

  Modifying Beamer's colours directly (rather than using a theme) requires you to use Beamer's framework.

  \onslide<2->
  The colour of any item in a Beamer presentation has a \alert<2>{\textbf{Beamer-color}}.

  \onslide<3->
  A Beamer-color has two parts:
  \begin{enumerate}
    \item<3-| alert@3> \texttt{fg}, foreground;
    \item<3-| alert@4> \texttt{bg}, background.
  \end{enumerate}

  \onslide<5->
  Like templates, these colours \alert<5>{\textbf{inherit}} attributes from parent colours.
  \begin{itemize}
    \item[e.g.] Changing the Beamer-color \texttt{structure} will alter many Beamer-colors, used by many objects in your presentation.
  \end{itemize}

\end{frame}

\begin{exercise}
Try adding the following (adapted from the manual, 186) to your preamble.
\begin{verbatim}
\setbeamercolor{structure}{fg=magenta!25!blue,bg=}
\setbeamercolor{item}{use={structure,normal text},fg=structure.fg!50!normal text.fg}
\end{verbatim}
If you want to use a Beamer-color directly, you can do so. What effect does adding
\begin{verbatim}
{\usebeamercolor[fg]{alerted text} test text}
\end{verbatim}
to the contents of a \textt{frame} have?
\end{exercise}


% END sec:colours


\mode
<all>

\furtherinfo*[%
  Beamer:
  \begin{itemize}
    \item Beamer's manual: \url{mirrors.ctan.org/macros/latex/contrib/beamer/doc/beameruserguide.pdf}.
    \item Appendix \ref{handouts:sec:simple}: a simple example to use as a base case.
    \item Appendix \ref{handouts:sec:builtin-themes}: lists of built-in themes for customisation.
    \item Appendix \ref{handouts:sec:overlay-macros}: creating custom overlay-aware commands and environments.
  \end{itemize}%
]
\mode*

\begin{frame}<0| article:1->{Beamer's Manual}
  \url{mirrors.ctan.org/macros/latex/contrib/beamer/doc/beameruserguide.pdf}
  \begin{itemize}
    \item All the gory details of Beamer's themes, colours, fonts, overlays, templates, environments, commands and options.
    \item Includes examples of recommended workflows for various scenarios, including the generation of supplementary materials and best practice for presentation design.
    \item Complete with colour illustrations.
  \end{itemize}
\end{frame}

\mode
<article>

% \clearpage
\vfill
\appendix

\section<1-| beamer:0>[A Simple Example]{A Simple Example\texorpdfstring{\footnote{\url{https://github.com/cfr42/latex-1-2/blob/master/gweithdy-latex-da-2-beamer/examples/example13.tex}}}{}}\label{sec:simple}
% BEGIN sec:simple
\verbatiminput{examples/example13}
\begin{figure}
  \centering
  \fbox{\includegraphics[page=1,width=.3\linewidth]{examples/example13}}
  \fbox{\includegraphics[page=2,width=.3\linewidth]{examples/example13}}
  \fbox{\includegraphics[page=3,width=.3\linewidth]{examples/example13}}\par
  \fbox{\includegraphics[page=4,width=.3\linewidth]{examples/example13}}
  \fbox{\includegraphics[page=5,width=.3\linewidth]{examples/example13}}
  \fbox{\includegraphics[page=6,width=.3\linewidth]{examples/example13}}
  \caption{A simple example (\cref{sec:simple}) typeset.}\label{fig:simple}
\end{figure}
The expected output is shown in \cref{fig:simple}.
% END sec:simple

\section<1-| beamer:0>{Built-In Themes}\label{sec:builtin-themes}
% BEGIN sec:builtin-themes

\subsection{Presentation Themes}\label{subsec:presentation-themes}

These themes typically specify a combination of a \texttt{color}, \texttt{font}, \textt{inner} and \texttt{outer} theme.

% BEGIN subsec:presentation-themes
\ttfamily
\begin{multicols}{4}\raggedcolumns
\begin{itemize}
  \item default
  \item boxes
  \item AnnArbor
  \item Antibes
  \item Bergen
  \item Berkeley
  \item Berlin
  \item Boadilla
  \item CambridgeUS
  \item Copenhagen
  \item Darmstadt
  \item Dresden
  \item EastLansing
  \item Frankfurt
  \item Goettingen
  \item Hannover
  \item Ilmenau
  \item JuanLesPins
  \item Luebeck
  \item Madrid
  \item Malmoe
  \item Marburg
  \item Montpellier
  \item PaloAlto
  \item Pittsburgh
  \item Rochester
  \item Singapore
  \item Szeged
  \item Warsaw
\end{itemize}
\end{multicols}
\normalfont
Most of these themes will have an immediate and dramatic effect on the appearance of your presentation.
However, \textt{boxes} is a little different from the others.
By default, it looks just like the \texttt{default} theme.
However, it adds two commands allowing you to create boxes in the headline and footline of slides.
The boxes are typeset in the order in which they are added and will be automatically adjusted to be of equal size.
\begin{itemize}
  \item \cs{addheadbox}\marg{\meta{beamer color}}\marg{\meta{box template}}
  \item \cs{addfootbox}\marg{\meta{beamer color}}\marg{\meta{box template}}
\end{itemize}
For example, suppose we add the following to the preamble of the example above\footnote{\url{https://github.com/cfr42/latex-1-2/blob/master/gweithdy-latex-da-2-beamer/examples/example14.tex}}.
\begin{verbatim}
\usetheme{boxes}
\addheadbox{structure}{First Box}
\addheadbox{section in head/foot}{Page \thepage}
\addfootbox{subsection in head/foot}{First Box}
\addfootbox{alerted}{Second Box}
\addfootbox{subsection in head/foot}{Third Box}
\end{verbatim}
\begin{figure}
  \centering
  \fbox{\includegraphics[page=1,width=.3\linewidth]{examples/example14}}
  \fbox{\includegraphics[page=2,width=.3\linewidth]{examples/example14}}
  \fbox{\includegraphics[page=3,width=.3\linewidth]{examples/example14}}\par
  \fbox{\includegraphics[page=4,width=.3\linewidth]{examples/example14}}
  \fbox{\includegraphics[page=5,width=.3\linewidth]{examples/example14}}
  \fbox{\includegraphics[page=6,width=.3\linewidth]{examples/example14}}
  \caption{Adding structure to the simple example in \ref{sec:simple} using a bundled theme.}\label{fig:theme-boxes}
\end{figure}
These additions add various structural elements to slides, as shown in \cref{fig:theme-boxes}.

% END subsec:presentation-themes

\subsection{Color Themes}\label{subsec:color-themes}

% BEGIN subsec:color-themes
\ttfamily
\begin{multicols}{4}\raggedcolumns
\begin{itemize}
  \item albatross
  \item beaver
  \item beetle
  \item crane
  \item default
  \item dolphin
  \item dove
  \item fly
  \item lily
  \item monarca
  \item orchid
  \item rose
  \item seagull
  \item seahorse
  \item sidebartab
  \item spruce
  \item structure
  \item whale
  \item wolverine
\end{itemize}
\end{multicols}
\normalfont
% END subsec:color-themes

\subsection{Font Themes}\label{subsec:font-themes}

% BEGIN subsec:font-themes
\ttfamily
\begin{multicols}{2}\raggedcolumns
\begin{itemize}
  \item default
  \item professionalfonts
  \item serif
  \item structurebold
  \item structureitalicserif
  \item structuresmallcapsserif
\end{itemize}
\end{multicols}
\normalfont
% END subsec:font-themes

\subsection{Inner Themes}\label{subsec:inner-themes}

% BEGIN subsec:inner-themes
\ttfamily
\begin{multicols}{4}\raggedcolumns
\begin{itemize}
 \item circles
  \item default
  \item inmargin
  \item rectangles
  \item rounded
\end{itemize}
\end{multicols}
\normalfont
% END subsec:inner-themes

\subsection{Outer Themes}\label{subsec:outer-themes}

% BEGIN subsec:outer-themes
\ttfamily
\begin{multicols}{4}\raggedcolumns
\begin{itemize}
  \item default
  \item infolines
  \item miniframes
  \item shadow
  \item sidebar
  \item smoothbars
  \item smoothtree
  \item split
  \item tree
\end{itemize}
\end{multicols}
\normalfont
% END subsec:outer-themes

% END sec:builtin-themes

\section<1-| beamer:0>{Creating Overlay-Aware Macros}\label{sec:overlay-macros}
% BEGIN sec:overlay-macros

Beamer extends the facilities of \cs{newcommand}, \cs{renewcommand}, \cs{newenvironment} and \cs{renewenvironment} (covered in \LaTeX{} II: Macros) to enable you to create your own overlay-aware macros.
\medskip
\par
\noindent
\cs{newcommand}{<>}\marg{\meta{new command name}}\oarg{\meta{argument number}}\oarg{\meta{default optional value}}\marg{\meta{definition}}
\par
\noindent
\cs{renewcommand}{<>}\marg{\meta{existing command name}}\oarg{\meta{argument number}}\oarg{\meta{default optional value}}
\marg{\meta{redefinition}}
\par
\noindent
\cs{newenvironment}{<>}\marg{\meta{new environment name}}\oarg{\meta{argument number}}\oarg{\meta{default optional value}}\marg{\meta{begin code}}\marg{\meta{end code}}
\par
\noindent
\cs{renewenvironment}{<>}\marg{\meta{existing environment name}}\oarg{\meta{argument number}}
\oarg{\meta{default optional value}}\marg{\meta{begin code}}\marg{\meta{end code}}
\medskip
\par
\noindent
The \meta{definition}, \meta{redefinition}, \meta{begin code} and \meta{end code} may use the place-holders \texttt{\#1},\dots,\texttt{\#n} where \texttt{n} is 1+\meta{argument number}.
\texttt{\#n} is then the optional overlay specification.

For example\footnote{\url{https://github.com/cfr42/latex-1-2/blob/master/gweithdy-latex-da-2-beamer/examples/example15.tex}},
\begin{verbatim}
  \newcommand<>{\sometimesbold}[1]{{\only#2{\bfseries}{#1}}}
  \sometimesbold<2>{Bold on the second slide.}
\end{verbatim}
will use bold only on the second slide.

\verbatiminput{examples/example15}
\begin{figure}
  \centering
  \fbox{\includegraphics[page=1,width=.3\linewidth]{examples/example15}}
  \fbox{\includegraphics[page=2,width=.3\linewidth]{examples/example15}}
  \caption{Effect of the custom overlay-aware macro defined in \cref{sec:overlay-macros}.}\label{fig:overlay-macros}
\end{figure}
The result is shown in \cref{fig:overlay-macros}.

% END sec:overlay-macros


\end{document}
