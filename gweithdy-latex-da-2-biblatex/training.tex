% BEGIN preamble
\svnidlong
{$HeadURL: file:///mnt/between/svn/dysgu/trunk/DoctoralAcademy/gweithdai-latex/gweithdy-latex-da-2-biblatex/training.tex $}
{$LastChangedBy: cfrees $}
{$LastChangedRevision: 7424 $}
{$LastChangedDate: 2018-03-09 00:17:21 +0000 (Gwe, 09 Maw 2018) $}

\input{config/macros-da.cfg}

\title{\LaTeX{} II: Bibliographies with Biber \& Biblatex}
\subtitle{Based in part on materials produced by UK-TUG Volunteers}
\date{\cymraeg{Mawrth} 20 March 2018}

% END preamble

\begin{document}

\begin{frame}
  \titlepage
\end{frame}

\maketitle

\tutornote{Time: 10:00}

\tableofcontents

\mode<presentation>%
{
  \begin{frame}{Outline}
    \tableofcontents
  \end{frame}
}

% Topics covered will include:
%
% Q referencing and bibliographies with Biblatex

\section{Introduction}

\begin{frame}{\LaTeX{}: Key features}

  \begin{itemize}
    \item \TeX{} is a typesetting application.
    \item \TeX{} uses \emph{primitives} to determine how to put text on a page.
    \item \LaTeX{} is  a \emph{format} built on top of \TeX{}.
    \item \LaTeX{} can be used for everything from a one page letter to a 1000~page textbook.
    \item By separating input from output, reusing material becomes much easier.
    \item By separating \emph{content} from \emph{formatting}, we can create flexible, consistent documents which are easier to maintain and modify.
    \item We are using the \emph{engine} pdf\TeX{}.
    Xe\TeX{} and Lua\TeX{} are modern alternatives.
  \end{itemize}

\end{frame}

\overleaf

\section{Biber and Biblatex}\tutornote{Time: 10:15}

\begin{frame}{Creating a bibliography}

  There are 3 common ways to create a bibliography in \LaTeX{}.
  \begin{itemize}
    \item Use the \env{thebibliography} environment.
    \item Use a database of bibliographical entries and \BibTeX{}.
    \item Use a database of bibliographical entries, \pkg{biblatex} and Biber.
  \end{itemize}

  The third option is the most flexible.
  \begin{itemize}
    \item Support for local and remote databases in different formats.
    \item Comprehensive support for online and electronic sources (e.g.~JSTOR).
    \item More logical (semantic) citation mark up.
    \item Built-in support for multiple bibliographies, multilingual typesetting and sophisticated sorting.
    \item Powerful and flexible customisation using \LaTeX{} macros.
  \end{itemize}

\end{frame}

\begin{frame}{Biber and Biblatex}

    \begin{itemize}
      \item Tell \LaTeX{} you are using Biblatex using \cs{usepackage}\oarg{backend=biber,\meta{other options}}\marg{biblatex}.
      \item Select citation and bibliography styles by choosing appropriate \meta{options}.
      \item Bibliography entries are stored in a \emph{database}.
      \item Most commonly, this is a \emph{\BibTeX{}} database.
      \item Inform \LaTeX{} about it using \cs{bibliography} or \cs{addbibresource}.
      \item These are cited using \cs{cite} in the \LaTeX{} file.
      \item Entries may be cited using an extended mark up.
      \item Print the bibliography using \cs{printbibliography}.
    \end{itemize}

\end{frame}

\begin{frame}[fragile]{Creating a bibliography}{Biblatex basics}

\begin{verbatim}
  \documentclass{article}
  \usepackage[utf8]{inputenc}
  \usepackage{csquotes,xpatch}
  \usepackage[backend=biber]{biblatex}
  \addbibresource{example}
  \begin{document}
    Some text \cite{key}.
    \printbibliography
  \end{document}
\end{verbatim}

\end{frame}

\begin{frame}{Biblatex workflow}

  \small\mode<article>{\footnotesize}
  \resizebox{\linewidth}{!}{%
    \begin{tikzpicture}[thick, >=Latex, font=\sffamily]
      \begin{scope}[minimum width=1.5cm,minimum height=1.5cm,inner sep=.1em]
        \path (0,0) node[gweithrediad] (edit) {\shortstack{Edit\\\texttt{.tex}\\File}};
        \path (0,2) node[gweithrediad] (editbib) {\shortstack{Edit\\\texttt{.bib}\\File}};
        \path (3.2,1) node[penderfyniad] (errors)
        {\shortstack{Any\\Errors?}};
        \path (6.8,1) node[penderfyniad] (bibtexerrors)
        {\shortstack{Any\\Errors?}};
        \path (10,1) node[penderfyniad] (errors2)
        {\shortstack{Any\\Errors?}};
        \path (12.5,1) node[canlyniad] (pdf) {\shortstack{View\\PDF}};
      \end{scope}
      \path (7,4.75) node[gweithrediad] (rmbbl) {Remove \texttt{.bcf} and/or \texttt{.bbl} file};
      \draw[->] (edit) -- (1,1) -- (errors) node[midway,above]
      {\shortstack{pdf\LaTeX{}\\\texttt{.tex} file}};
      \draw (editbib) -- (1,1);
      \draw[->] (errors) -- ++(0,-2.25) node[midway,left] {Yes}
      -| (edit) node[pos=0,below,anchor=north east]
      {Error in document};
      \draw[->] (errors) -- (bibtexerrors)
      node[midway,above] {\shortstack{No errors.\\Run Biber}};
      \draw[->] (bibtexerrors) -- ++(0,-3)
      node[midway,right] {Yes} -| (edit)
      node[pos=0,below,anchor=north east]
      {Misspelt/missing label or bib style};
      \draw[->] (bibtexerrors) -- ++(0,2.5) node[midway,right]
      {Yes} -| (editbib) node[pos=0,above,anchor=south east]
      {Error in bib file};
      \draw[->] (bibtexerrors) -- (errors2) node[midway,above]
      {\shortstack{No errors.\\Run pdf\LaTeX{}\\(twice)}};
      \draw[->] (errors2) |- (rmbbl) node[pos=0.2,right] {Yes};
      \draw[->] (rmbbl) -| (editbib);
      \draw[->] (errors2) -- (pdf) node[midway,above] {No};
    \end{tikzpicture}%
  }

\end{frame}

\begin{exercise}\label{ex:biblatex-example}\tutornote{Ref.: \cref{sec:samples}.}
  This exercise is designed to give you a sense of what Biblatex/Biber can do.
  Rather than creating your own database of bibliography entries, you will use an example database supplied with the \pkg{biblatex} package.

  Create a file called, say, \texttt{example11.tex} that contains the following: \verbatiminput{examples/example11.tex}

  Overleaf should automatically compile your document to include the information from your database.
  If you were using a terminal or command prompt, you would need to use the following commands:
  \begin{verbatim}
    pdflatex example11
    biber example11
    pdflatex example11
    pdflatex example11
  \end{verbatim}

  There are various options you can pass to the \pkg{biblatex} package that affects the formatting.
  For example, try changing the options you give Biblatex as follows:
  \begin{verbatim}
    \usepackage[style=authoryear-comp,mergedate=basic,isbn=false,url=false,%
    eprint=true,doi=false,sortcites=true,backend=biber,mincrossrefs=6]{biblatex}
  \end{verbatim}
  Try experimenting with some of these options:
  \begin{itemize}
    \item \texttt{style=alphabetic},
    \item \texttt{style=authoryear-comp},
    \item \texttt{style=numeric},
    \item \texttt{style=numeric-comp}.
  \end{itemize}
  What effect does \texttt{sorting=none} have?
\end{exercise}

\begin{frame}[fragile]{A \BibTeX{} database}{A basic article}\tutornote{Time 11:00}

\begin{example}\tutornote{\raggedright Comment on other database formats.}
  \begin{semiverbatim}
  \alert<2>{@book}\alert<3>{\{}\alert<4>{lamport94}\alert<5>{,}
    \alert<6>{author}    \alert<7>{= \{}\alert<8>{Leslie Lamport}\alert<7>{\},}
    \alert<6>{title}     \alert<7>{=
    \{}\alert<8-9>{\{\\LaTeX\{\}\}}\alert<8>{: A Document Preparation System}\alert<7>{\},}
    \alert<6>{edition}   \alert<7>{= \{}\alert<8>{2}\alert<7>{\},}
    \alert<6>{publisher} \alert<7>{= \{}\alert<8>{Addison-{}-Wesley}\alert<7>{\},}
    \alert<6>{address}   \alert<7>{= \{}\alert<8>{New York}\alert<7>{\},}
    \alert<6>{date}      \alert<7>{= \{}\alert<8,10>{{1994}}\alert<7>{\},}
  \alert<3>{\}}
  \end{semiverbatim}
\end{example}

\end{frame}

\begin{frame}[fragile]{A \BibTeX{} database}{Multiple authors}

\begin{example}
  \begin{semiverbatim}
  @inproceedings\{smith05,
    author    = \{\alert<4-5>{Smith, Jr, John} \alert<2>{and} \alert<3>{Jane Lucy Doe}
    \alert<2>{and} \alert<4>{Other, Andrew N.} \alert<2>{and} \alert<4>{de Vere, Jo}\},
    title     = \{An Example Article\},
    booktitle = \{Proceedings of the Imaginary Society\},
    date      = \{\alert<6>{2005-01}\}
  \}
  \end{semiverbatim}
\end{example}

\end{frame}

\begin{frame}{Citations using Biblatex}

  \begin{block}{Textual citations}
    \vspace*{0.5 em}
    \begin{tabular}{l@{\ $\Rightarrow$\ }l}
      \multicolumn{2}{l}{%
        \alert<1>{\cs{textcite}\oarg{\meta{note}}\marg{\meta{key}}}%
      } \\[0.5em]
      \onslide<2->
      \cs{textcite}\marg{lamport1994} & Lamport (\meta{label}) \\
      \cs{textcite}\oarg{34}\marg{lamport1994} &
      Lamport (\meta{label}, p.~34) \\
    \end{tabular}
  \end{block}

  \onslide<3->
  \begin{block}{Parenthetical citations}
    \vspace*{0.5 em}
    \begin{tabular}{l@{\ $\Rightarrow$\ }l}
      \multicolumn{2}{l}{%
        \alert<3>{\cs{parencite}\oarg{\meta{prenote}}\oarg{\meta{postnote}}%
        \marg{\meta{key}}%
      }}  \\[0.5em]
      \onslide<4->
      \cs{parencite}\marg{lamport94} & (\meta{label})\\
      \cs{parencite}\oarg{\textasciitilde34}\marg{lamport94}
      & (\meta{label}, p.~34)\\
      \cs{parencite}\oarg{see}\oarg{}\marg{lamport94} & (see \meta{label})
    \end{tabular}
  \end{block}

\end{frame}

\begin{frame}{Citations using Biblatex}\tutornote{Note facilities for custom citation commands.}

  \begin{block}{Citations in footnotes}
    \alert<1>{\cs{footcite}\oarg{\meta{prenote}}\oarg{\meta{postnote}}%
        \marg{\meta{key}}}
  \end{block}

  \onslide<2->
  \begin{block}{Automatic formatting}
    \alert<2>{\cs{autocite}\oarg{\meta{prenote}}\oarg{\meta{postnote}}%
        \marg{\meta{key}}}
  \end{block}

\end{frame}


\begin{exercise}\label{ex:database}
  Create a file called \texttt{myrefs.bib} that contains the following: \verbatiminput{examples/myrefs.bib}

  Then create a file called, say, \texttt{example12.tex} that contains the following: \verbatiminput{examples/example12.tex}

  Try adding the following to \texttt{myrefs.bib}:
  \verbatiminput{examples/myrefs-add.bib}

  Now add \verb|\Textcite{smullyan-unfortunate-dualist-minds-i}| to \texttt{example12.tex}.
  What gets added to the list of references?

\end{exercise}

\furtherinfo*[%
  Biblatex/Biber:
  \begin{itemize}
    \item Appendix \ref{handouts:sec:samples}: sample output for exercises \ref{handouts:ex:biblatex-example} and \ref{handouts:ex:database}.
    \item Appendix \ref{handouts:sec:pellach}: a very selective introduction to a small part of the additional functionality provided by Biblatex/Biber.
  \end{itemize}%
]

\appendix\label{app}
\section<1-| beamer:0>{Sample Output}\label{sec:samples}
% BEGIN sec:samples

This appendix includes sample output for the code in \cref{ex:biblatex-example,ex:database}.

For \cref{ex:biblatex-example} output is shown first with just \texttt{backend=biber} (the default) and then with the options \texttt{style=authoryear-comp, mergedate=basic, isbn=false, url=false, eprint=true, doi=false, sortcites=true, backend=biber, mincrossrefs=6}.

Likewise, output for \cref{ex:database} is shown first with the initial references and then with the suggested additions.

\includepdf{examples/example11-1}
\includepdf{examples/example11-2}
{\centering
  \fbox{\includegraphics[width=.95\textwidth,height=.95\textheight,keepaspectratio]{examples/example12-1}}\par
  \fbox{\includegraphics[width=.95\textwidth,height=.95\textheight,keepaspectratio]{examples/example12-2}}\par
}

% END sec:samples

\section<1-| beamer:0>{Further Exploration}\label{sec:pellach}
% BEGIN sec:pellach

\subsection{Electronic Resources}\label{subsec:eres}
% BEGIN subsec:eres

As seen in the exercises, Biblatex provides support for electronic resources in a variety of ways.
Most simply, it can include a URL or DOI.
However, these are often long and unwieldy, cluttering bibliographies with long strings of awkward characters.
Moreover, many kinds of electronic resource have an identifier of some kind, which uniquely locates the source according to a systematic schema.

JSTOR provides identifiers of this kind.
As shown in the entry \verb|nagel-death|, Biblatex can use JSTOR's identifier if specified.
Nagel's paper is located at \url{http://www.jstor.org/stable/2214297}.
\texttt{2214297} identifies the source uniquely.
Biblatex knows about JSTOR, so including the lines
\begin{verbatim}
  eprint = {2214297},
  eprinttype = {jstor},
\end{verbatim}
in the entry enables Biblatex to typeset the identifier.
If \pkg{hyperref} is loaded, this identifier will be linked to the relevant URL automatically.
Hence, there is actually no need to include the URL explicitly in the entry.

This is extremely useful: identifiers can be used to automatically link to the relevant locations on the web, without the need to include the URL in either the database or the typeset bibliography.

However, this only works if an \texttt{eprinttype} is known to Biblatex.
Inevitably, Biblatex has rather limited knowledge of electronic resources.
By default, it knows about only \texttt{arXiv} (\textt{arxiv}), \textsc{jstor} (\texttt{jstor}), PubMed (\texttt{pubmed}), \textsc{hdl} (\texttt{hdl}) and Google Books (\texttt{googlebooks}).

The easiest way to configure support for custom resource types is to add suitable definitions to a personal \textt{biblatex.cfg}.
This file should be located in your personal \textsc{texmf} tree, in sub-directory \texttt{tex/latex/config/}.

On a Unix-type system you can create the directory like this:
\begin{verbatim}
mkdir -p $(kpsewhich -var TEXMFHOME)/tex/latex/config
\end{verbatim}
Then create a file with the following content and save it as \texttt{biblatex.cfg} in that directory.
\begin{verbatim}
\ProvidesFile{biblatex.cfg}

% add support for Project Gutenberg identifiers
\DeclareFieldFormat{eprint:gutenberg}{%
  Project\space Gutenberg\space ebook\addcolon\space
  \ifhyperref
  {\href{http://www.gutenberg.org/ebooks/#1}{\nolinkurl{#1}}}
  {\nolinkurl{#1}}}
\DeclareFieldAlias{eprint:GUTENBERG}{eprint:gutenberg}

\endinput
\end{verbatim}
Further details, including ways to accommodate more complex cases, can be found in Biblatex's manual.

% END subsec:eres

\subsection{Multiple Bibliographies}\label{subsec:multibib}
% BEGIN subsec:multibib

Biblatex/Biber supports multiple bibliographies in a single document, subject to certain limitations.
Two basic environments are provided to accommodate split bibliographies.
The first is \env{refsection} and the second is \env{refsegment}.

The two commonest cases are probably the need to provide a separate bibliography in each section of a document and the need to split a bibliography according to some characteristic of the entries such as primary \emph{vs.}\ secondary sources.

If you want a bibliography per chapter or section or other document division, add the option
\begin{semiverbatim}
  refsection=\meta{division}
\end{semiverbatim}
\meta{division} may be one of \texttt{part}, \texttt{chapter}, \texttt{section} or \texttt{subsection}.
Then use \cs{printbibliography} in each part, chapter, section or subsection to produce a bibliography specific to that division or the document.

If you want to print out all the bibliographies together at the end of the document, use \cs{bibbysection} to loop over them all one-by-one.

\env{refsection} makes the labels in your bibliography local to the relevant part of the document.
So, if you use numeric labels, for example, reference \texttt{19} may refer to a text by Aristotle in chapter 1 and a play by Goethe in chapter 2.

If you want labels to be unique so that Goethe gets a different label from Aristotle, use the \env{refsegment} environment instead.
\begin{semiverbatim}
  refsegment=\meta{division}
\end{semiverbatim}
\meta{division} may be one of \texttt{part}, \texttt{chapter}, \texttt{section} or \texttt{subsection}.
Then use \cs{bibbysegment} at the end of the document to print all segments' bibliographies one-by-one.

Further details, options and usage examples are included in \pkg{biblatex}'s manual.

% END subsec:multibib

\subsection{Maintenance}\label{subsec:maintain}

% BEGIN subsec:maintain

If you update your \TeX{} distribution, you will probably get errors when you compile a document using Biblatex/Biber.
This is because the package is actively developed and the current code is often not compatible with the previous code.
In this case, delete all generated bibliography-related files, especially the \texttt{.bcf}, \texttt{bbl} and \texttt{.blg}.
Then recompile, rerun Biber and compile again.

If you update your \TeX{} distribution using your distribution's tools, the versions of Biblatex and Biber should be kept automatically in sync.
Should you for any reason want to try a development version of Biber or Biblatex, note that you \textbf{MUST} use a pair whose versions are designed to work together.
Biblatex's documentation includes a list of Biber versions required for particular versions of Biblatex.
You need to ensure that you have a matching pair or you will simply get errors.

% END subsec:maintain

% END sec:pellach

\end{document}
